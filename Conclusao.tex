
O autor do trabalho apresentou a base teórica dos dispositivos APS e TIA, além das especificações de um projeto de Receptor Óptico, as soluções propostas pelo autor para o seu desenvolvimento. as simulações necessárias para avaliação da proposta, e a elaboração do layout dos principais circuitos dentro do trabalho.

Diversos desafios apareceram ao longo de todo o trabalho, dos quais necessitam de ampla pesquisa acadêmica e também orientação para o devido entendimento dos problemas que apareceram.

O trabalho como um todo teve seus objetivos alcançados, mostrando o desenvolvimento e simulação de um projeto de Receptor Óptico para três cores, e seu circuito integrado foi desenvolvido para que seja realizado a sua validação.

\subsection{Trabalhos futuros}

O autor propõe que com a fabricação do circuito integrado, sejam realizadas todas as medições para validação do trabalho, e que este possa servir de conhecimento para outros que se proporem em estender ao que foi aqui desenvolvido.

Para as simulações, algumas estimativas foram realizadas pelo autor, que com as amostras do CI podem ser aprimoradas, com foco em trabalhos futuros. Simulações mais precisas de forma a gerar modelos e circuitos mais complexos podem ser realizados com base no que foi aqui apresentado.

As Figuras de Mérito apresentadas podem ser utilizadas para comparar as características do circuito aqui apresentado com outros presentes na literatura.