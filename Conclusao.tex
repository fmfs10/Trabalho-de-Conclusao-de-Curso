
O autor do trabalho apresentou a base teórica dos dispositivos aqui apresentados, as especificações de um projeto de Receptor Óptico, as soluções propostas para o seu desenvolvimento, as simulações necessárias para avaliação da proposta, e a elaboração do layout dos principais circuitos dentro do trabalho.

O projeto como um todo necessitou de um esforço em conjunto de várias pessoas para que pudesse ser realizado, que foram citadas ao longo do trabalho assim que apresentados suas atividades.

Diversos desafios apareceram ao longo de todas o trabalho, dos quais necessitaram de ampla pesquisa acadêmica e também orientação para o devido entendimento dos problemas que apareceram.

O trabalho como um todo teve seus objetivo alcançado, mostrando o desenvolvimento e simulação de um projeto de Receptor Óptico para três cores, e seu circuito integrado foi desenvolvido para que seja realizado a sua validação.

\subsection{Trabalhos futuros}

O autor propõe que com a fabricação do circuito integrado seja realizado as medições para validação do trabalho, e que possa servir de conhecimento para outros que se proporem em estender ao que foi aqui desenvolvido.

Para as simulações, algumas estimativas foram realizadas pelo autor, que com as amostras do CI podem ser aprimoradas. Simulações mais precisas de forma a gerar modelos e circuitos mais complexos podem ser realizados com base no que foi aqui apresentado.