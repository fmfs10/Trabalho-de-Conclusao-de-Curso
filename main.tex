\include{Configuracoes/ConfiguracoesABNT}

% ---
% Informações de dados para CAPA e FOLHA DE ROSTO
% ---
\titulo{Projeto de Circuitos Analógicos para Optoreceptores Integrados CMOS}
\autor{Felipe Magalhães Fonseca Silva}
\local{Belo Horizonte, Minas Gerais, Brasil}
\data{8 de março de 2021}
\orientador{Dalton Martini Colombo}
\coorientador{}
\instituicao{%
  Universidade Federal de Minas Gerais - UFMG
  \par
  Escola de Engenharia
  \par
  Departamento de Engenharia Elétrica
 }
\tipotrabalho{Monografia (Graduação)}
% O preambulo deve conter o tipo do trabalho, o objetivo, 
% o nome da instituição e a área de concentração 
\preambulo{Monografia de conclusão de curso para obtenção do nível de Bacharel em Engenharia Elétrica pela Escola
de Engenharia da Universidade
Federal de Minas Gerais.}
% ---

\include{Configuracoes/ConfiguracoesAparencia}

% ----
% Início do documento
% ----
\begin{document}

% Seleciona o idioma do documento (conforme pacotes do babel)
%\selectlanguage{english}
\selectlanguage{brazil}

% Retira espaço extra obsoleto entre as frases.
\frenchspacing 

% ----------------------------------------------------------
% ELEMENTOS PRÉ-TEXTUAIS
% ----------------------------------------------------------
% \pretextual

% ---
% Capa
% ---
\imprimircapa
% ---

% ---
% Folha de rosto
% (o * indica que haverá a ficha bibliográfica)
% ---
\imprimirfolhaderosto*
% ---

\input{FichaCatalografica}

% ---
% Dedicatória
% ---
\begin{dedicatoria}
   \vspace*{\fill}
   \centering
   \noindent
   \textit{ Este trabalho é dedicado aos meu queridos pais,\\
   pessoas essenciais para toda minha formação\\
   e meus maiores exemplos de vida} \vspace*{\fill}
\end{dedicatoria}
% ---

\input{FolhaAprovacao}

% ---
% Agradecimentos
% ---
\begin{agradecimentos}
Gostaria de agradecer a todos aqueles que contribuíram, direta ou indiretamente, para que este trabalho fosse possível.

Meus pais, que sempre foram exemplos para mim de carinho, dedicação e respeito. Eu definitivamente não chegaria onde cheguei sem vocês. Os considero as pessoas mais importantes da minha vida.

Meus irmãos, que sempre me incentivaram a crescer e a evoluir, de forma que eu possa apreciar com muito mais riqueza todos os momentos de felicidade que já tive e que terei daqui pra frente.

Meus amigos, que sempre me ajudaram a seguir essa longa jornada chamada vida, trazendo momentos de felicidade e companheirismo, mesmo nos instantes em que pensava não ser mais possível de os alcançar. Obrigado, de verdade, por me mostrarem que sempre terei pessoas com quem eu possa confiar e compartilhar todos meus momentos de alegria e tristeza.

Todos meus professores, que foram essenciais para a minha formação e a apreciar a maravilhosa e assustadora Engenharia Elétrica.

Meu orientador, Dalton Martini Colombo, que se tornou não só minha referência de formação como também de vida. Os incentivos que recebi durante todo o percurso, e tudo que você propiciou para mim durante este processo, será algo que lembrarei sempre com muito carinho e admiração.

Ao OptMa\textsuperscript{Lab}, pois o trabalho aqui apresentado não envolveu um esforço único, mas de um conjunto de pessoas extremamente competentes que se dedicaram para que esse projeto pudesse acontecer.

\end{agradecimentos}
% ---

% ---
% Epígrafe
% ---
\begin{epigrafe}
    \vspace*{\fill}
	\begin{flushright}
		\textit{Eu não sou o senhor do tempo, mas eu sei que vai chover\\
        Me sinto muito bem quando fico com você\\
        Eu tenho habilidade de fazer histórias tristes\\
        Virarem melodia, vou vivendo o dia a dia}
        
        \textit{
        Na paz, na moral, na humilde, busco só sabedoria\\
        Aprendendo todo dia, me espelho em você\\
        Corro junto com você, vivo junto com você\\
        Faço tudo por você}
        
        (Charlie Brown Jr. - Senhor do Tempo)
	\end{flushright}
\end{epigrafe}
% ---

% ---
% RESUMOS
% ---

% resumo em português
\setlength{\absparsep}{18pt} % ajusta o espaçamento dos parágrafos do resumo
\begin{resumo}

O trabalho explora a implementação de Optoreceptores utilizando a tecnologia CMOS de 180 nm da TSMC. São estudados e desenvolvidos quatro APS’s (Active Sensor Pixel), além de dois Amplificadores de Transimpedância (TIA) contendo um Fotodiodo em cada.
Três APS’s tem a função de captação de uma cor distinta (Azul, Verde e Vermelho) de uma fonte luminosa. O circuito desenvolvido é feito de maneira idêntica para as 3 cores, e um filtro de luz externo é utilizado para que seja captada apenas a cor de interesse. O APS é constituído por um fotodiodo quadrado de dimensões 25x25 µm (625 µm²) e um circuito para captação da corrente fotogerada e envio de informação referente a luz em sua saída. Na saída do APS, o sinal é comparado à uma tensão de referência e amplificado, permitindo flexibilização na utilização do circuito para diferentes intensidades de fotocorrente. O sinal então caminha para um multiplexador que seleciona a saída que irá ao pino do Circuito Integrado, podendo ser qualquer um dos APS’s.
Um quarto APS é utilizado apenas para fins de teste. A constituição física deste componente é idêntica aos outros, com exceção de um pino extra do qual pode ser injetada uma tensão diretamente ao fotodiodo, de forma a permitir a simulação de corrente fotogerada mesmo sem a presença de luz.
O TIA tem a função de extrair um sinal de referência de uma fonte luminosa e servir como base para os outros circuitos de forma a gerar temporizações a todos APS’s. O fotodiodo tem dimensão de 25x25 µm (625 µm²), e também um amplificador com tensão de referência com o mesmo uso do APS. O circuito foi projetado de forma a permitir uma frequência de pelo menos 100 kHz. Assim como o APS de teste, existe um segundo TIA com um pino extra para simular uma fotocorrente sem a pŕesença de luz.
Um chip de 1,6x1,6 mm (2,56 mm²) em encapsulamento CLC44 foi projetado, contendo os circuitos aqui citados como também outros circuitos de interesse do laboratório OptMA. Todo o projeto foi disponibilizado para fabricação em setembro de 2020, e sua caracterização será possível a partir de março de 2021.


 \textbf{Palavras-chave}: latex. abntex. editoração de texto.
\end{resumo}


% resumo em inglês
\begin{resumo}[Abstract]
 \begin{otherlanguage*}{english}

This work explores the implementation of Optoreceptors using a \textit{CMOS TSMC 180 nm} technology from TSMC. An APS (Active Sensor Pixel) and a Transimpedance Amplifier (TIA) are proposed. Three APSs have the function of capturing different colors (Blue, Green or Red) from a light source. The developed circuit is made identically for the 3 colors, and an external light filter is used so that only the color of interest is captured. The APS consists of a square photodiode of dimensions 25x25$\mu$m (625$\mu$m²), associated with a circuit for converting the photogenerated current to an output voltage. At the APS output, the signal is compared to a reference voltage during its digitalization. Reference voltage is programmable, allowing flexibility in digitalizing different intensities of photogenerated current. An APS is used only for testing purposes and its electrical design is the same as the others, with the exception of an extra pin from which an electric current that emulates the photoconductive effect can be drained, without the presence of light. The TIA has the function of extracting a clock reference signal from a light source, at the same time as it serves as a time reference for other circuits, for example, APSs pixels. The photodiode used has a dimension of 25x25$\mu$m (625$\mu$m²). The circuit was designed to allow a nominal operating frequency of 100 kHz. Like the test APS, there is a second TIA with an extra pin to simulate a photocurrent without the presence of light. The layout design for several analog circuits and auxiliary mixed signals blocks related to the operation of the APS and the TIA devices were also made on this work. The complete integrated circuit occupies an area of 1.6x1.6 mm (2.56 mm²), as it also contains a voltage oscillator, temperature sensor, and microLEDS actuators. The project was submitted for production in September 2020, and its characterization will be possible from March 2021.

   \vspace{\onelineskip}
 
   \noindent 
   \textbf{Keywords}: APS. CMOS. PHOTODIODE. TIA.
 \end{otherlanguage*}
\end{resumo}

% ---

% ---
% inserir lista de ilustrações
% ---
\pdfbookmark[0]{\listfigurename}{lof}
\listoffigures*
\cleardoublepage
% ---

% ---
% inserir lista de tabelas
% ---
\pdfbookmark[0]{\listtablename}{lot}
\listoftables*
\cleardoublepage
% ---

\include{ListaSiglas}

\include{ListaSimbolos}

% ---
% inserir o sumario
% ---
%\pdfbookmark[0]{\contentsname}{toc}
\tableofcontents*
\cleardoublepage
% ---

% ----------------------------------------------------------
% ELEMENTOS TEXTUAIS
% ----------------------------------------------------------
\textual
% ----------------------------------------------------------
% Introdução (exemplo de capítulo sem numeração, mas presente no Sumário)
% ----------------------------------------------------------
\chapter[Introdução]{Introdução}
%$\addcontentsline{toc}{chapter}{Introdução}
% ----------------------------------------------------------
Receptores ópticos têm uma vasta gama de aplicações. Podemos utilizá-los para detectar a presença ou ausência de luz no ambiente em determinada faixa de frequência; recepção de sinais luminosos de diferentes cores para construir uma foto em uma câmera; recepção de sinais codificados em luz para realizar uma transmissão de dados sem fio (Li-Fi); detecção de temperatura em um objeto via a sua emissão de luz infravermelha; entre incontáveis outras aplicações. A forma como o circuito do sensor é desenvolvido varia muito com a aplicação, e diversas topologias podem ser utilizadas. Mesmo dentro de uma classe em específico de sensores, diversas variações de circuitos podem existir.

Neste trabalho, é desenvolvido o projeto eletrônico dos principais blocos analógicos de um receptor óptico integrado em tecnologia CMOS 180 nm. Para realizar a recepção dos dados, um pixel ativo do tipo APS (Active Pixel Sensor) foi projetado. É um circuito utilizado para situações em que não só a presença, mas a intensidade de determinada frequência óptica é de interesse ao usuário. O APS já é uma topologia muito estudada e utilizada, devido principalmente à facilidade com que pode ser integrado em um CI utilizando tecnologia CMOS, facilidade de acoplamento com outros circuitos do mesmo CI, e também pelo seu baixo custo por pixel em comparação com outros tipos de Receptores Ópticos. 
Junto à construção do circuito do APS, diversos outros blocos de circuitos são necessários para que se permita o condicionamento e processamento do sinal luminoso transferido para o domínio elétrico. Circuitos de controle devem ser desenvolvidos para garantir sincronia entre todos elementos do sistema. Dentre os circuitos auxiliares que compõem todo um sistema de Pixel Ativo, podemos citar Amplificadores Operacionais, utilizados para amplificar o sinal gerado pelo sensor; multiplexadores, para realizarem a seleção do sinal de diferentes pixeis de luz ou outros componentes do sistema; e circuitos geradores de clock, que definem o momento de captação de luz do pixel e momento este poderá ser operado.

O desenvolvimento dos principais circuitos analógicos de um receptor óptico integrado serão apresentados, desde o seu detalhamento teórico, concepção do circuito e subcircuitos necessários ao sistema, desenvolvimento de esquemático para realização de simulações iniciais, construção do layout de circuito integrado,, desenvolvimento de um CI para permitir a ligação dos sinais e realização de testes, e finalmente, medições para validação de todo sistema. A tecnologia utilizada no processo é a CMOS TSMC 180nm.
Discussões sobre os resultados obtidos são realizadas, com o intuito de averiguar as variações do modelo teórico e de simulação com aquilo que foi devidamente fabricado. Alinhado às discussões, é feita uma averiguação de possíveis melhorias para futuros projetos para otimização de todo o sistema em diversos fatores.

Todo o trabalho é desenvolvido com apoio do OptMALab , laboratório que se apresenta internamente à Escola de Engenharia da UFMG, e que possui o ferramental necessário para permitir a realização de todos o procedimentos aqui citados, com exceção da fabricação do APS e seu encapsulamento, que é realizada em local externo por meio de uma parceria da universidade com a foundry Interuniversity Microelectronics Centre (IMEC); e da fabricação do PCB, feito por uma empresa externa.

\section{Motivação e Objetivos do Trabalho}

Dado a infinidade de aplicações do qual um APS pode ser utilizado, a fundamentação de parâmetros de referência para desenvolvimento de projetos de circuitos que se faz presente é de essencial importância para possibilitar a escolha de qual projeto é mais adequado à aplicação. Diferentes figuras de mérito podem ser avaliadas para otimizar uma ou mais características de interesse técnico e/ou econômico do projeto.

O trabalho tem como principal objetivo o desenvolvimento de um circuito de Sensor de Pixel Ativo, e um TIA, tomando como referência trabalhos j\'a desenvolvidos por outros autores. O desenvolvimento apresenta o projeto elétrico, simulação, e desenvolvimento do layout do CI. 

São desenvolvidos seis circuitos principais, sendo três com a função de detectar diferentes cores (azul, verde e amarelo) utilizando um APS, um circuito extrator de sinal de relógio de uma fonte luminosa atrav\'es de um TIA, e um APS e um TIA para fins de testes, dos quais s\~ao utilizados para testar as funcionalidades dos circuitos citados sem a necessidade de uma fonte luminosa. 

O dispositivo projetado deve se adequar a todas as características recomendáveis de um projeto de circuito analógico integrado CMOS, devendo ser avaliado e otimizado para aprimorar certas propriedades importantes de layout como perdas ohmicas, capacitâncias parasitas, ruídos e consumo.

Alguns circuitos auxiliares compõem conjuntamente o projeto, como Amplificadores Operacionais, Filtros e Multiplexadores, que são elementos comuns em projetos de APS’s e TIA's, utilizados para transferir e controlar a informação gerada pelo sensor, mas que não são o principal foco de análise.

\section{Estrutura do Trabalho}

A monografia apresenta o desenvolvimento de um projeto de APS's e TIA's com base em algumas figuras de mérito comumente utilizadas na indústria.. O texto foi dividido em quatro capítulos, além desta introdução, sendo eles Revisão Bibliográfica, Metodologia, Resultados e Conclusão.

Revisão Bibliográfica, capítulo 2 deste trabalho, visa apresentar um resumo histórico simplificado sobre Receptores Ópticos, conceituação de elementos chave para o entendimento do que é apresentado em capítulos posteriores, citação de outros trabalhos relacionados ao tema aprovados pela comunidade científica e detalhamento do trabalho base que foi tomado como principal inspiração da monografia.
Metodologia, capítulo 3 deste trabalho, visa apresentar o processo adotado para desenvolvimento do circuito de interesse, com base no trabalho de inspiração citado, demonstrando o processo de projeto à nível de esquemático elétrico, simulações, avaliações de figuras de mérito em ambiente simulado, design do layout do circuito analógico, design do PCB para utilização do CI projetado e a realização de medições dos parâmetros do CI após sua devida fabricação.

Resultados, capítulo 4 deste trabalho, apresenta todos resultados obtidos de acordo com o modelo metodológico previamente apresentado, e a comprovação de atendimento aos objetivos tais quais formulados. Uma série de imagens, gráficos e dados são apresentados de forma a permitir a avaliação do projeto como um todo.

Conclusão, capítulo 5 deste trabalho, retoma os resultados e avalia se os objetivos foram devidamente cumpridos. Sugestões de trabalhos posteriores são apresentados para explorar alguns tópicos que o autor entende que possam complementar todo o estudo aqui realizado.

\section{Padroniza{\c c}\~ao de identifi{\c c}\~ao dos elementos dos circuitos}
\label{section:padrao_sinais}

Para facilitar a demonstra{\c c}\~ao dos circuitos presentes neste trabalho, o autor padronizou um conjunto de cores que estar\~ao presentes em cada elemento ilustrados para cada circuito referenciado:

\begin{itemize}
    \item \textbf{\color{red}VERMELHO}: indica o nome de uma inst\^ancia de um componente
    \item \textbf{\color{orange}LARANJA}: indica o nome de um sinal interno
    \item \textbf{\color{green}VERDE}: indica um sinal de entrada
    \item \textbf{\color{blue}AZUL}: indica um sinal de sa\'ida
    \item \textbf{\color{gray}CINZA}: indica o nome de um sinal bidirecional (entrada e/ou sa\'ida)
\end{itemize}
% ----------------------------------------------------------
% Introdução (exemplo de capítulo sem numeração, mas presente no Sumário)
% ----------------------------------------------------------
\chapter[Revisão Bibliográfica]{Revisão Bibliográfica}
%$\addcontentsline{toc}{chapter}{Introdução}
% ----------------------------------------------------------

A literatura sobre Receptores Ópticos \'e extensa e contempla os mais diversos aspectos sobre o tema, desde o entendimento do fenômeno físico, diferentes topologias e estrat\'egias para otimização de diferentes características do circuito, circuitos auxiliares necessários para a transferência de informação, e tamb\'em o layout do projeto. Este capítulo tem como objetivo contextualizar e fundamentar algumas discussões de capítulos posteriores, demonstrar alguns trabalhos previamente realizados na área, e dar uma perspectiva histórica sobre algumas tecnologias envolvidas.
\section{Histórico}
O interesse de extrair informações advindas do espectro eletromagn\'etico \'e a muito tempo de grande interesse para físicos e engenheiros. As suas aplicações são diversas, desde extrair cores para formar uma imagem at\'e extrair informação digital de acordo com um sinal luminoso modulado.
Em 1949, a primeira patente relacionada a um dispositivo semicondutor capaz de transformar informações luminosas em el\'etricas foi requisitada pela Bell Labs, sendo oficialmente aceita em 1951. Era o nascimento do fototransistor, um transistor bipolar especial que capta informação luminosa em sua base, gerando uma informação por forma de uma corrente el\'etrica em seu coletor e emissor \cite{Shive}.

Com invenção do fototransistor e fotodiodo, e o desenvolvimento de tecnologias de fabricação de semicondutores, se gerou uma intensa pesquisa na d\'ecada de 60, que culminou no desenvolvimento do CCD (\emph{Charge Coupled Device}) no início da d\'ecada de 70, sendo a tecnologia de Recepção Óptica mais utilizada at\'e a d\'ecada de 90 \cite{EstevaoCoelho, Andre}.

Em meados da d\'ecada de 70, o PPS (\emph{Passive Pixel Sensor}) foi desenvolvido, como alternativa aos dispositivos de imagem de tubo a vácuo \cite{Savvas}. Sua constituição se dá por uma matriz de fotodiodos, que convertem uma informação luminosa em el\'etrica, diretamente a um elemento processador de dados analógico, sem a utilização de um componente de amplificação.

Ainda na d\'ecada de 70, os primeiros trabalhos relacionados ao \emph{APS} foram desenvolvidos utilizando a tecnologia \emph{MOS} (\emph{Metal-Oxide-Semiconductor}) \cite{Peter}. O \emph{APS} \'e uma evolução direta do \emph{PPS} (\emph{Passive Pixel Sensor}), com a adição de um dispositivo amplificador \cite{EstevaoCoelho}. Amplificar o sinal permite uma maior precisão na aquisição dos dados, al\'em de permitir uma maior miniaturização do fotodiodo, o que leva a uma miniaturização de todo o sistema. Com a evolução da tecnologia CMOS (\emph{Complementary Metal-Oxide-Semiconductor}) na d\'ecada de 90, o \emph{APS} se tornou mais atraente, por ser facilmente integrado com outros componentes que utilizam-se da mesma tecnologia em um mesmo circuito integrado. O \emph{APS} vem desde então dominando o mercado de Recepção Óptica \cite{Andre, LidianeCampos}.

Amplificadores Operacionais s\~ao circuitos amplamente utilizados para o desenvolvimento de sistemas analógicos. Desde a década de 60, o desenvolvimento de circuitos de comunicação óptica exploram a utilização de amplificadores para captação de sinais ópticos, transmissão e regeneração de sinais. A utilização de amplificadores operacionais junto à fotodiodos, possibilitam a conversão de sinais de luz em elétricas, que pode ser então processado e utilizado para fins tanto analógicos quanto digitais. O circuito comumente utilizado no processo de captação óptica são variações do \emph{TIA}, tendo papel fundamental para a expansão tecnológica que vemos tendo desde então. \cite{ajoy, andrefontoura}.

\section{Fotodiodo}

O fotodiodo, ou fotodetector, \'e um componente optoeletrônico semicondutor que tem a função de captar um sinal eletromagn\'etico no ambiente, e então convertê-lo para uma corrente el\'etrica, chamada de corrente fotogerada, ou tamb\'em fotocorrente \cite{RazaviOpt}. A faixa de frequência do sinal que pode ser captado varia em cada projeto, mas dispositivos práticos se apresentam comumente em uma banda entre o ultravioleta e o infravermelho, o que possibilita o desenvolvimento de dispositivos dentro da faixa da luz visível \cite{LidianeCampos}.

Para o entendimento do funcionamento do fotodiodo, uma fundamentação quântica da luz deve ser utilizada, com foco principal na Teoria das Bandas e no entendimento do que são os Fótons.

\subsection{Física Quântica e Fotóns}
A Física Quântica nos diz que qualquer material apresenta níveis de energia possíveis dados de forma discreta (quantizada), ou seja, qualquer objeto do universo não pode ter qualquer nível de energia, mas sim valores dados de forma discreta, e que são quantificados por meio da solução da \textit{Equação de Schrödinger} do material \cite{Sze, JohnSingleton}.


\subsubsection{Teoria das Bandas}

A Teoria das Bandas nasce da observação de que comumente uma grande quantidade de níveis de energia permitidos de um material se apresentam muito próximos entre si. Desses níveis de energia, surge-se o conceito “banda de energia”, que \'e a aproximação que dentro dessa faixa de níveis concentrados, temos uma representação contínua de níveis de energia permitidos. Um material pode apresentar uma grande quantidade de bandas, sendo que duas são de extremo interesse para o estudo de semicondutores: A Banda de Condução e a Banda de Valência.

A Banda de Valência \'e a última do qual se considera que um el\'etron está fortemente ligada a um átomo, não podendo mover livremente ao longo do material. A Banda de Condução \'e a primeira banda de níveis acima da de Valência, e nele já consideramos que um el\'etron \'e livre para se movimentar no material.

\subsubsection{Fóton}

Fóton \'e uma partícula elementar que surge como resultado da liberação de energia ocasionada pela migração de um el\'etron da Banda de Condução para o de Valência. Este \'e o elemento fundamental constituinte da luz, e pela dualidade onda-partícula, pode ser visto tanto do ponto de vista de uma partícula, que se movimenta no espaço e pode chocar com um material; quanto de onda movimentando no espaço, apresentando uma frequência formada pela oscilação de dois campos (El\'etrico e Magn\'etico), que \'e regida pelas Leis de Maxwell.

\subsection{Fotodiodo}
\label{secao_fotodiodo}
Um fotodiodo comumente \'e formado pela junção PIN, que \'e a junção de um material semicondutor dopado do tipo-P, um semicondutor intrínseco e um material semicondutor dopado do tipo-N, como mostrado na \autoref{fig_fotodiodo}:

\begin{figure}[htb]
	\caption{\label{fig_fotodiodo}Exemplo de construção de uma junção PIN}
	\begin{center}
	    \includegraphics[scale=0.4]{Imagens/PIN.png}
	\end{center}
	\legend{Fonte: \cite{RazaviOpt}}
\end{figure}

Quando um fóton com energia suficiente para excitar o material \'e absorvido, um par el\'etron-lacuna \'e gerado (migração de um el\'etron da banda de valência para de condução), e por difusão, os el\'etrons migram para o cátodo, formando uma fotocorrente. Ajustando-se os tamanhos das camadas P, N e da região de depleção (formada pelo semicondutor intrínseco), podemos controlar a resposta em frequência do fotodiodo e qual a energia mínima do fóton necessária para que gere a fotocorrente. Com a migração dos el\'etrons, uma diferença de potencial \'e gerada entre as camadas P e N, que pode ser aproveitada para a geração de corrente el\'etrica em um circuito el\'etrico fechado \cite{hamamatsu}.
Um fotodiodo apresenta um modelo el\'etrico equivalente ilustrado na \autoref{fig_modelofotodiodo}:

\begin{figure}[htb]
	\caption{\label{fig_modelofotodiodo}Modelo el\'etrico de um fotodiodo}
	\begin{center}
	    \includegraphics[scale=0.5]{Imagens/ModeloFotodiodo.png}
	\end{center}
	\legend{Fonte: \cite{hamamatsu}}
	\label{modeloElFotodiodo}
\end{figure}

    O modelo el\'etrico apresenta a seguinte express\~ao:

\begin{equation}
    \label{eq_modEletFot}
    I_o = I_L - I_D - I\rq = I_L - I_S*(\exp (\frac{qV_d}{kT})-1) - I\rq
\end{equation}

Onde:
\begin{itemize}
    \item \emph{I\textsubscript{o}} \'e a corrente de saída presente na carga [\emph{A}]
    \item \emph{V\textsubscript{o}} \'e a tensão de saída [\emph{V}]
    \item \emph{R\textsubscript{L}} \'e a carga de saída [\emph{$\Omega$}]
    \item \emph{V\textsubscript{D}} \'e a tens\~ao presente no diodo do modelo el\'etrico equivalente [\emph{V}]
    \item $I_L$ \'e a corrente fotogerada pela fonte luminosa[\emph{A}]
    \item  \emph{I\textsubscript{D}} \'e a corrente corrente de escuro do fotodiodo (sem a presença de luz) [\emph{A}]
    \item \emph{C\textsubscript{j}} \'e a capacit\^ancia de junç\~ao [F]
    \item \emph{R\textsubscript{sh}} \'e a resist\^encia shunt do modelo el\'etrico equivalente [$\Omega$]
    \item \emph{I\rq} \'e a corrente shunt presente na resistência shunt do modelo el\'etrico equivalente [A]
    \item \emph{R\textsubscript{S}} \'e a resistência em s\'erie com a carga de saída do modelo el\'etrico equivalente [$\Omega$]
    \item \emph{q} \'e a carga el\'etrica de um el\'etron [C], \item \emph{k} \'e a constante de Boltzmann [J.K$^{-1}$],
    \item \emph{T} \'e a temperatura presente no fotodiodo [K]
\end{itemize}

O fotodiodo em uma grande faixa apresenta comportamento linear, e \'e a faixa de principal interesse em aplica{\c c}\~oes como o $APS$ ou o $TIA$. Com o aumento do n\'umero de f\'otons recebidos, o n\'ivel de corrente tende a ficar cada vez mais negativo, como mostrado na \autoref{fig_respFotodiodo}. Com o aumento ou diminui{\c c}\~ao expressiva, o circuito come{\c c}a a se tornar n\~ao-linear.

\begin{figure}[htb]
	\caption{\label{fig_respFotodiodo}Resposta Tens\~ao x Corrente para dada intensidade luminosa no fotodiodo}
	\begin{center}
	    \includegraphics[scale=0.8]{Imagens/graficoRespostaFotodiodo.png}
	\end{center}
	\legend{Fonte: \cite{hamamatsu}}
\end{figure}

\subsection{Figuras de M\'erito}
Do fotodiodo podemos extrair diversas m\'etricas (Figuras de M\'erito), que são de interesse para comparar diferentes modelos. As principais Figuras de M\'erito encontradas na literatura são descritas abaixo, apresentados em \cite{LidianeCampos}.

\subsubsection{Eficiência Quântica}
Relação entre o número de portadores detectados nos terminais das camadas PN do fotodetector, dividido pela incidência de uma determinada quantidade de fótons no fotodetector.

\begin{equation}
    \eta = \frac{N_e}{N_p}
\end{equation}

Onde:
\begin{itemize}
    \item \emph{$\eta$} \'e Efici\^encia Qu\^antica [Adm.]
    \item \emph{N$_e$} \'e o n\'umero de portadores que podem ser detectados nos terminais externos do fotodetector [n° portadores]
    \item \emph{N$_p$} incid\^encia de determinada quantidade de f\'otons [n° f\'otons]
\end{itemize}

\subsubsection{Responsividade}
Razão entre a corrente fotogerada e a potência óptica incidida no fotodiodo.

\begin{equation}
    \label{eq_responsividade}
    R_\lambda = \frac{I_{PH}}{P_{FD}}
\end{equation}

Onde:
\begin{itemize}
    \item \emph{R$_\lambda$} \'e a Responsividade [A.W$^{-1}$]
    \item \emph{I$_{PH}$} \'e a corrente fotogerada [A]
    \item \emph{P$_{FD}$} \'e a pot\^encia \'optica presente no fotodiodo [W]
\end{itemize}

A Eficiência Quântica e a Responsividade se relacionam de acordo com a \autoref{eqEfResp}.

A \autoref{fig_eqEfResp} mostra o gráfico da Responsividade e Efici\^encia Qu\^antica de alguns materiais.

\begin{equation}
    \label{eqEfResp}
    R_\lambda = \frac{\lambda\eta}{1,24}
\end{equation}

Onde:
\begin{itemize}
    \item \emph{R$_\lambda$} \'e a Responsividade [A.W$^{-1}$]
    \item $\lambda$ \'e o comprimento de onda da luz incidente [m]
    \item $\eta$ \'e Efici\^encia Qu\^antica [Adm.]
\end{itemize}

\begin{figure}[htb]
	\caption{\label{fig_responsividade}Responsividade e Efici\^encia Qu\^antica dos materiais Ge, InGaAs, Si}
	\begin{center}
	    \includegraphics[scale=0.5]{Imagens/GraficoRespostaEspectral.png}
	\end{center}
	\legend{Fonte: \cite{ajoy}}
	\label{fig_eqEfResp}
\end{figure}


\subsubsection{Velocidade de Resposta}
A velocidade de resposta indica o quão rápido o fotodiodo \'e capaz de responder a estímulos de luz externos, em determinada frequência. \'e caracterizado pelo Tempo de Subida e Tempo de Descida do fotodiodo, na faixa de frequências de interesse.

O Tempo de Subida \'e calculado como o tempo do qual um fotodiodo, inicialmente sem incidência de luz, leva para elevar o seu nível em tensão nos seus terminais de 10\% para 90\% do seu pico, a partir do no momento que começar a absorver fótons de uma fonte luminosa controlada, em determinada frequência.

O Tempo de Descida \'e calculado como o tempo do qual um fotodiodo, inicialmente com determinada incidência de luz e já estabilizado na sua respectiva tensão de pico, leva para diminuir a diferença de tensão entre seus terminais de 90\% para 10\% do seu pico, a partir do momento que não absorver mais fótons de origem de uma fonte luminosa controlada, em determinada frequência.

\begin{figure}[htb]
	\caption{\label{fig_velocidadeResp}Representação gr\'afica do Tempo de Subida e Descida}
	\begin{center}
	    \includegraphics[scale=0.3]{Imagens/GraficoVelocidadeResposta.png}
	\end{center}
	\legend{Fonte: Produzido pelo autor}
\end{figure}

O conhecimento da velocidade de resposta \'e crucial para determinar as limitações do projeto em termos de processamento de informações. A amostragem de dados deve ter período maior do que os tempos de subida e de descida do fotodiodo, nos maiores valores apresentados na faixa de frequência desejada.

\subsubsection{Resposta Espectral}
	O Efeito Fotoel\'etrico se caracteriza pela excitação de el\'etrons em um material por um fóton, quando este apresenta uma energia mínima, chamada de Função Trabalho. Mesmo quando uma grande quantidade de fótons com energia menor do que a Função Trabalho atravessem o fotodiodo, eles não serão capazes de excitar nenhum el\'etron. Como a energia de um fóton diminui com o aumento de seu comprimento de onda, existe um comprimento de onda máximo (ou uma frequência mínima) do qual o fotodetector será capaz de gerar fotocorrente.
	Por outro lado, uma diminuição do comprimento de onda tende a excitar cada vez menos el\'etrons, pois  absorção na região de depleção deixa de acontecer, chegando a um ponto que o fotodiodo sature em resposta e não seja mais possível diferenciar diferentes frequências acima desta.
Sabendo destas limitações, definimos a Resposta Espectral como a faixa de comprimentos de onda do qual o fotodetector \'e capaz de produzir uma fotocorrente correspondente.

\subsubsection{Relação Sinal-Ruído}
Razão entre a potência de sinal fotogerada pela pot\^encia de ru\'ido no sinal.

\begin{equation}
    SNR = \frac{P_{SINAL}}{P_{RUÍDO}}
\end{equation}

Onde:
\begin{itemize}
    \item \emph{SNR} \'e relação sinal-ru\'ido [Adm.]
    \item \emph{P$_{SINAL}$} \'e a pot\^encia do sinal fotogerado [W]
    \item \emph{V\textsubscript{D}} \'e a pot\^encia do ruido do sinal fotogerado [W]
\end{itemize}

\subsubsection{Pot\^encia equivalente ao ru\'ido}
Pot\^encia da luz incidida no fotodiodo, que gera uma potência de sinal equivalente ao de ruído em largura de 1 Hz

\begin{equation}
    NEP = P_{Ru\'ido 1 Hz}
\end{equation}

Onde:
\begin{itemize}
    \item \emph{NEP} \'e a pot\^encia da luz incidida no fotodiodo [W]
    \item \emph{$P_{Ruído 1 Hz}$} \'e a pot\^encia do ruído em largura de 1 Hz
\end{itemize}

\subsubsection{Detectividade espec\'ifica}
Razão entre a raiz quadrada da área fotossensível do diodo, sobre o NEP

\begin{equation}
    D* = \frac{\sqrt{A.B}}{NEP}
\end{equation}

Onde:

\begin{itemize}
    \item \emph{D*} \'e a detectividade espec\'ifica [$cm.\sqrt{Hz}.W^{-1}$]
    \item \emph{A} \'e a \'area da regi\~ao fotossens\'ivel do fotodiodo [cm²]
    \item \emph{A} \'e a largura de banda [Hz]
\end{itemize}

\section{Sensor de Pixel Ativo (APS)}
\label{section:APS}
Um APS \'e um dispositivo do qual se aproveita das características de um fotodiodo para gerar um sinal que pode ser amostrado e então quantificado, de forma a produzir informações referentes à luz incidida no fotodetector.

Dentre as várias possibilidades de produção de um circuito APS, aquele estudado ao longo de todo este trabalho se apresenta na \autoref{fig_APS}.

\begin{figure}[htb]
	\caption{\label{fig_APS}Circuito APS do trabalho}
	\begin{center}
	    \includegraphics[scale=0.3]{Circuitos/APS.png}
	\end{center}
	\legend{Fonte da ilustra{\c c}\~ao: Pr\'oprio autor}
\end{figure}

\begin{figure}[htb]
	\caption{\label{fig_APS_block}Representa{\c c}\~ao do circuito APS em bloco}
	\begin{center}
	    \includegraphics[scale=0.3]{Circuitos/APS_block.png}
	\end{center}
	\legend{Fonte: Produzido pelo autor}
\end{figure}

Cada componente desempenha uma diferente fun{\c c}\~ao de forma a processar a informação advinda da corrente fotogerada:

\begin{itemize}

    \item O transistor \emph{$T_{buffer}$} funciona como um amplificador de Dreno Comum, e seu papel \'e replicar o sinal advindo do n\'o central ao n\'o de sa\'ida, reduzindo o efeito de carga que aconteceria caso o n\'o central ($V_{cn}$) fosse a saída do sistema.

    \item O transistor \emph{$T_{reset}$} funciona como uma chave, que quando fechada (sinal 0 em seu gate), vale \emph{VDD}. Quando aberto (sinal 1 em seu gate), o sinal no n\'o central passa a depender da configuração do transmition gate.

    \item O transmition gate \emph{$T_{enable}$} funciona como uma chave, e quando o transistor \emph{$T_{reset}$} estiver em aberto, tem a função controlar se a tensão presente no fotodiodo ir\'a para o nó central ou n\~ao. Quando aberto, $V_{pn}$ fica isolado do restante do circuito, devido ao estado de alta imped\^ancia nos terminais do dispositivo. Quando fechado, a corrente fotogerada descarrega em $V_{cn}$, pois a associação das capacitâncias parasitas presentes nos transistores de \emph{$T_{enable}$}, \emph{$T_{reset}$} e \emph{$T_{buffer}$} formam um caminho fechado do qual descarrega lentamente o fotodiodo \cite{LidianeCampos}. Uma representa{\c c}\~ao desses capacitores \'e dada na \autoref{fig_APS_cap}.
    
    \item O transistor \emph{$T_{select}$} \'e utilizado para que m\'ultiplos APS's compartilhem um mesmo barramento de sa\'ida. No projeto aqui implementado \emph{$T_{select}$} sempre se apresentar\'a fechado (\emph{SELECT} em n\'ivel l\'ogico '1').
    
    \item A fonte de corrente \emph{$I_{ref}$} tem a fun{\c c}\~ao de polarizar a sa\'ida do circuito, além de aprimorar a linearidade do estágio de sa\'ida \cite{RazaviFundM}.

\end{itemize}

    H\'a dois n\'os internos de bastante interesse ao trabalho, que s\~ao:

\begin{itemize}
    \item \emph{$V_{pn}$}: tens\~ao entre os terminais do fotodiodo
    \item \emph{$V_{cn}$}: tens\~ao no n\'o central do \emph{APS}
\end{itemize}

    Os sinais externos do circuito s\'ao:
    
\begin{itemize}
    \item \emph{RESET}: fecha o transistor \emph{$T_{reset}$}. Ativo em n\'ivel l\'ogico '0'.
     \item \emph{SELECT}: fecha o transistor \emph{$T_{select}$}. Ativo em n\'ivel l\'ogico '1'. No circuito do trabalho estar\'a sempre configurado como '1'.
     \item \emph{ENABLE}: fecha o transmition gate \emph{$T_{enable}$}. Ativo em n\'ivel l\'ogico '1'.
     \item \emph{VDD}: Alimenta{\c c}\~ao do circuito
\end{itemize}

\subsubsection{Capacit\^ancias Parasitas}
Para o correto estudo do \emph{APS} devemos observar as capacit\^ancia equivalente presentes nos n\'os \emph{$V_{pn}$} e \emph{$V_{cn}$} do circuito \cite{LidianeCampos}. Na \autoref{fig_APS_cap} temos uma representa{\c c}\~ao dessas capacit\^ancias.

\begin{figure}[htb]
	\caption{\label{fig_APS_cap}Representa{\c c}\~ao do circuito APS com suas capacit\^ancias parasitas destacadas}
	\begin{center}
	    \includegraphics[scale=0.3]{Circuitos/APS_cap.png}
	\end{center}
	\legend{Fonte: Produzido pelo autor}
\end{figure}

\clearpage

Onde: 

\begin{itemize}
    \item $C_j$ \'e a capacit\^ancia de jun{\c c}\~ao do fotodiodo. Seu principal efeito no estudo do APS \'e limitar a velocidade de varia{\c c}\~ao da corrente fotogerada.
    \item $C_{cn}$ \'e a capacit\^ancia equivalente vista do n\'o \emph{$V_{cn}$} ao terra, devido principalmente \'as capacit\^ancias presentes em \emph{$T_{enable}$},  \emph{$T_{reset}$} e \emph{$T_{buffer}$}, sendo o de \emph{$T_{buffer}$} o de maior contribui{\c c}\~ao. O capacitor cria um caminho fechado para a corrente do fotodiodo circular ao terra e ser descarregado em um dos est\'agios citados posteriormente no trabalho.
    \item \emph{$I_{ph}$} \'e a corrente fotogerada
    \item \emph{Diode} \'e o diodo do modelo el\'etrico equivalente do fotodiodo
    \emph{}
\end{itemize}

\subsubsection{Est\'agios}
\label{estagiosAPS}

A operação do APS pode ser dividida em 4 est\'agios, que vão definir os m\'inimos momentos de  atua{\c c}\~ao no sinal de controle e seus limites opera{\c c}\~ao. Os est\'agios s\~ao representados na \autoref{figura_estagiosAPS}, tendo como base o trabalho de \cite{LidianeCampos}.

\begin{figure}[htb]
	\caption{\label{figura_estagiosAPS}Esbo{\c c}o de resposta do APS ao controlar seus sinais de entrada}
	\begin{center}
	    \includegraphics[scale=0.2]{Imagens/estagiosAPS.png}
	\end{center}
	\legend{Fonte: Adaptado de \cite{LidianeCampos}}
\end{figure}

\begin{enumerate}

\item \emph{$T_{reset}$} fechado, \emph{$T_{enable}$} fechado
    
Nessa condição, o n\'o $V_{cn}$ \'e igual a VDD.

\item \emph{$T_{reset}$} aberto, \emph{$T_{enable}$} fechado

A corrente fotogerada passa a descarregar a capacit\^ancia $C_{cn}$ do nó central, e então pelas capacitância parasita \emph{$C_{cn}$}. A tensão no fotodiodo passa a diminuir, devido a circulação de corrente que diminui a carga de \emph{$C_{j}$}.

A tens\~ao nos terminais do fotodiodo neste segundo est\'agio tem uma rela{\c c}\~ao linear com a corrente fotogerada, conforme descrito na \autoref{eq_modEletFotIl}, que \'e deduzida fazendo an\'alises dos n\'os, desprezando-se a resistência presente em $T_{enable}$ e tamb\^em outra resistências parasitas, como aquelas presentes no modelo elétrico do fotodiodo apresentado na \autoref{modeloElFotodiodo}.

\begin{equation}
    \label{eq_modEletFotIl}
    V(t) = VDD-\frac{I_{PH}}{C_j+C_{cn}}t
\end{equation}

Onde:

\begin{itemize}
    \item $VDD$ \'e a tens\~ao de alimenta{\c c}\~ao do circuito [$V$]
    \item $V$ \'e a tens\~ao presente no terminal de catodo do fotodiodo (\autoref{secao_fotodiodo}) [$V$]
    \item $I_{PH}$ \'e a corrente fotogerada [$A$]
    \item $C_j$ \'e a capacit\^ancia de jun{\c c}\~ao do fotodiodo [$F$]
    \item $C_{cn}$ \'e a capacit\^ancia do n\'o central do APS [$F$]
    \item $t$ \'e o tempo [$s$]
\end{itemize}

\item \emph{$T_{reset}$} aberto, \emph{$T_{enable}$} aberto, potencial \emph{$V_{pd}$} positivo

Logo ap\'os abrir o $T_{enable}$, a fotocorrente n\~ao circula mais no n\'o central, e o fotodiodo continua a ter sua diferen{\c c}a de potencial reduzida, com circula{\c c}\~ao de corrente internamente devido a \emph{$C_j$}. O n\'o central passa a cair bem lentamente devido a correntes de fuga. \emph{$V_{pd}$} cai at\'e chegar a se tornar negativo;

\item \emph{$T_{reset}$} aberto, \emph{$T_{enable}$} aberto, potencial \emph{$V_{pd}$} negativo

Nessa situação, o $T_{enable}$ passa a operar em modo linear. O potencial no n\'o $V_{cn}$ passa a cair devido a circulação de corrente do n\'o at\'e o fotodiodo. O nó diminui at\'e que finalmente chega a 0, onde se mant\'em estável e não apresenta mais circula{\c c}\~ao de corrente.

\end{enumerate}

\subsection{Amostrando informações da luz com um APS básico}

Podemos aproveitar o entendimento das propriedades f\'isicas do fotodiodo, e tamb\'em dos est\'agios de funcionamento de um APS, para obtermos informações relativas aos f\'otons absorvidos.

Como sabemos matematicamente as relações de fotocorrente determinadas por \autoref{eq_modEletFot} e \autoref{eq_modEletFotIl}, e tamb\'em as Figuras de M\'erito que caracterizam o fotodetector, podemos utilizar o APS para coletar as informações em sua sa\'ida e determinar a intensidade da luz

Entre os est\'agios 2 e 3, a tens\~ao do n\'o $V_{cn}$ passa a cair, devido a circula{\c c}\~ao da fotocorrente. Como essa corrente depende da intensidade da luz, podemos descobrir qual o valor de intensidade sabendo a inclina{\c c}\~ao da curva de tens\~ao na sa\'ida, j\'a que a varia{\c c}\~ao da tens\~ao \'e uma grandeza diretamente proporcional a corrente e a imped\^ancia vista no n\'o. Como a tens\~ao de sa\'ida \'e igual ao de $V_{cn}$ menos $V_{GS}$ do transistor $T_{buffer}$, podemos medir a sa\'ida para processar o sinal e ent\~ao relatar a intensidade luminosa.

Com as observa{\c c}\~oes aqui apresentadas, podemos desenvolver um sistema de medi{\c c}\~ao de informa{\c c}\~ao luminosa, trabalhando entre os est\'agios 1 e 3, e ent\~ao retornando ao 1 para uma nova aquisi{\c c}\~ao. \'E importante destacar que existem limitações quanto \`a temporiza{\c c}\~ao dos est\'agios. Para que a medi{\c c}\~ao seja realizada de forma adequada, devemos garantir que tenhamos entre o Est\'agio 1 e 2, um tempo suficientemente grande para que $V_{cn}$ apresente um valor estável, ou seja, o tempo de transi{\c c}\~ao do sistema nessa condi{\c c}\~ao seja conclu\'ido. Entre o Est\'agio 2 e Est\'agio 3, devemos garantir que tenhamos um tempo mínimo para que dois valores distintos de $V_{out}$ possam ser medidos, de acordo com a sensibilidade do sistema de medi{\c c}\~ao.

\section{Amplificador de Transimped\^ancia (TIA)}
\label{section:TIA}

Um amplificador de transimped\^ancia \'e um circuito em que dada uma corrente de entrada, gera-se uma tens\~ao em sua sa\'ida proporcional \`a esta corrente \cite{RazaviFundM}. Considerando-se o fotodiodo como uma fonte de corrente, a \autoref{fig_TIA} apresenta uma poss\'ivel topologia de um \emph{TIA}, desenvolvido no presente trabalho.

\begin{figure}[htb]
	\caption{\label{fig_TIA}\emph{TIA} desenvolvido}
	\begin{center}
	    \includegraphics[scale=0.3]{Circuitos/TIA.png}
	\end{center}
	\legend{Fonte da ilustra{\c c}\~ao: Pr\'oprio autor}
\end{figure}

Onde:

\begin{itemize}
    \item $V_{ref}$ \'e a tens\~ao de entrada (ou $refer\^encia$) [$V$]
    \item $V_o$ \'e a tens\~ao de sa\'ida [$V$]
    \item $R$ \'e uma resist\^encia de ajuste ganho [$\Omega$]
    \item $R_l$ \'e uma carga de sa\'ida [$\Omega$]
\end{itemize}

Sabemos da \autoref{fig_modelofotodiodo} que podemos modelar uma resist\^encia $R_{sh}$ em paralelo ao fotodiodo. Desprezando-se todos outros componentes do modelo para facilitar a an\'alise, podemos descrever a f\'ormula correspondente ao $TIA$ como a \autoref{eqCTIA}.

\begin{equation}
    \label{eqCTIA}
    V_o = RI_{PH} + (1+\frac{R}{R_{sh}})V_{ref}
\end{equation}

Onde:
\begin{itemize}
    \item $I_{PH}$ \'e a corrente fotogerada [$A$]
\end{itemize}

A \autoref{eqCTIA} nos mostra que \emph{R} deve ser ajustado de forma que n\~ao seja grande demais, pois valores altos (de acordo com a aplicação) irão gerar tens\~ao na sa\'ida com varia{\c c}\~ao muito grande. Mesmo que \emph{Vref} fosse colocado em \emph{GND}, o fator $(1+\frac{R}{R_{sh}})$ ainda pode dominar a relação, devido \'a ruídos presentes no nó, e então impor um termo indesejado, o que também causaria problemas quando \emph{R} fosse muito alto \cite{hamamatsu}.

Devido ao produto $RI_{PH}$, também devemos ter o cuidado de não fazer \textit{R} pequeno demais para que tenhamos a sensibilidade no sinal de saída de forma adequada.

\subsection{Resposta espectral de um \emph{TIA}}

Como a imped\^ancia de entrada do $TIA$ n\~ao \'e ideal (infinita) e varia com a frequ\^encia e temperatura, o circuito n\~ao se apresenta linear em toda sua banda de opera{\c c}\~ao, o que representa uma varia{\c c}\~ao na resposta em frequ\^encia de acordo com a corrente fotogerada \cite{hamamatsu}.
Como o pr\'oprio amplificador apresenta capacit\^ancias internas, o circuito \'e tamb\'em limitado em altas frequ\^encias, principalmente pelo produto $R$ vezes $C$, onde C \'e a capacit\^ancia interna vista pelos terminais de $R$ \cite{hamamatsu}.

Dada as duas propriedades, a \autoref{figura_respostaTIA} e a \autoref{figura_respostaTIA2} mostram o comportamento t\'ipico de resposta em frequ\^encia da topologia.

\begin{figure}[htb]
	\caption{\label{figura_respostaTIA}Resposta espectral de um TIA}
	\begin{center}
	    \includegraphics[scale=0.8]{Imagens/RespostaEspectralTIA.png}
	\end{center}
	\legend{Fonte: \cite{hamamatsu}}
\end{figure}

\begin{figure}[htb]
	\caption{\label{figura_respostaTIA2}Resposta a um pulso em um TIA}
	\begin{center}
	    \includegraphics[scale=0.8]{Imagens/RespostaEspectralTIA2.png}
	\end{center}
	\legend{Fonte: \cite{hamamatsu}}
\end{figure}

\clearpage

% ----------------------------------------------------------
% PARTE
% --------------------------------------------------------
% ----------------------------------------------------------
Um circuito integrado foi desenvolvido para avaliar cinco dispositivos APS's, referenciado na se{\c c}\~ao \ref{section:APS} deste documento, e dois TIA's, em refer\^encia \`a se{\c c}\~ao \ref{section:TIA}. O projeto \'e representado em alto n\'ivel na \autoref{fig_circcompleto}. Todos os sinais aqui apresentados na figura, e tamb\'em em todas figuras \'a seguir, seguem o a padr\~o descrito na se{\c c}\~ao \ref{section:padrao_sinais} deste trabalho.

\begin{figure}[htb]
	\caption{\label{fig_circcompleto}Circuito projetado}
	\begin{center}
	    \includegraphics[width=\textwidth]{Circuitos/Complete_Circuit.png}
	\end{center}
	\legend{Fonte: Produzido pelo autor}
\end{figure}

O circuito tem a finalidade de processar a informa{\c c}\~ao advinda de tr\^es APS's, constru\'idos de maneira id\^entica em n\'ivel de layout, com a finalidade de abstrair informa{\c c}\~oes de cores advindas de uma fonte luminosa, dos quais podem ser Azul, Verde ou Vermelha. Para que as cores fossem devidamente separadas entre cada APS, filtros luminosos externo, por meio de uma pel\'icula colorida, s\~ao adicionados sob cada APS respons\'avel por processar uma cor equivalente \`a de sua pel\'icula.

Um sinal luminoso de cor branca tamb\'em \'e utilizado no sistema de forma a ser a refer\^encia de rel\'ogio de todos APS's descritos. Esse sinal \'e processado utilizando-se um $TIA$, do qual \'e gerado um sinal el\'etrico equivalente \'a informa{\c c}\~ao luminosa.

A tecnologia de fabrica{\c c}\~ao utilizada para o desenvolvimento de todos blocos foi a \emph{CMOS 180nm da TSMC}. O software utilizado para o projeto do dispositivo foi o \emph{Virtuoso}, desenvolvido pela \emph{Cadence}.

O circuito representado na \autoref{fig_circcompleto} \'e composto por 2 blocos principais, que permitem o processamento advindos da fonte luminosa, al\'em de um circuito \emph{APS} e um \emph{TIA} extra. A descri{\c c}\~ao dos bloco s\~ao:

\begin{itemize}
    \item \emph{ibias\_block}: Tem a fun{\c c}\~ao de gerar todas fontes de corrente utilizadas em todos os blocos do circuito, quando necess\'ario.
    
    \item \emph{4\_APS}: Implementa os tr\^es circuitos APS descritos, al\'em do circuito \emph{TIA}. A sa\'ida de cada bloco passa por um comparador de forma a digitalizar o dado, como ser\'a melhor explicitado na se{\c c}\~ao \autoref{Bloco4APS}.
    
    \item \emph{Vref\_block} e \emph{PIXEL\_CLK\_TEST}: Estes blocos realizam a implementa{\c c}\~ao de um TIA, por\'em com a adi{\c c}\~ao de um pino extra que possibilita a simula{\c c}\~ao de uma corrente fotogerada sem necessitar de uma fonte luminosa. Estes blocos ser\~ao melhor explicados na se{\c c}\~ao \autoref{BlocoTIA}. 
    
    \item \emph{Vref\_block} e \emph{PIXEL\_TEST}: Este blocos realizam a implementa{\c c}\~ao de um APS, por\'em com a adi{\c c}\~ao de um pino extra que possibilita a simula{\c c}\~ao de uma corrente fotogerada sem necessitar de uma fonte luminosa. Estes blocos ser\~ao melhor explicados na se{\c c}\~ao \autoref{BlocoAPS}.
    
\end{itemize}

A \autoref{tab_circcomp} mostra a rela{\c c}\~ao de sinais de entrada e sa\'ida presentes no circuito, para o processamento dos p\'ixels de cor. A \autoref{tab_circcomp2} mostra a rela{\c c}\~ao de sinais de entrada e sa\'ida presentes no circuito para o processamento dos blocos de teste.

\begin{table}[htb]
\IBGEtab{%
  \caption{Descri{\c c}\~ao dos sinais de entrada e sa\'ida do circuito projetado para as cores azul, verde e vermelha}%
  \label{tab_circcomp}
}{%
  \begin{tabular}{ccll}
  \toprule
   Sinal & Tipo & Descri{\c c}\~ao & Observa{\c c}\~ao \\
  \midrule \midrule
   RESET\_BLUE & Entrada & Sinal de \emph{RESET} no APS para cor azul & Ativo em n\'ivel baixo \\
  \midrule
   RESET\_GREEN & Entrada & Sinal de \emph{RESET} no APS para cor verde & Ativo em n\'ivel baixo \\
  \midrule
   RESET\_RED & Entrada & Sinal de \emph{RESET} no APS para cor vermelha & Ativo em n\'ivel baixo \\
  \midrule
   ENABLE\_BLUE & Entrada & Sinal de \emph{ENABLE} no APS para cor azul & Ativo em n\'ivel alto \\
  \midrule
   ENABLE\_GREEN & Entrada & Sinal de \emph{ENABLE} no APS para cor verde & Ativo em n\'ivel alto \\
  \midrule
   ENABLE\_RED & Entrada & Sinal de \emph{ENABLE} no APS para cor vermelha & Ativo em n\'ivel alto \\
  \midrule
   Out\_An\_Blue & Sa\'ida & Sinal anal\'ogico para cor azul \\
  \midrule
   Out\_Dig\_Blue & Sa\'ida & Sinal digital para cor azul \\
  \midrule
   Out\_An\_Green & Sa\'ida & Sinal anal\'ogico para cor verde \\
  \midrule
   Out\_Dig\_Green & Sa\'ida & Sinal digital para cor verde \\
  \midrule
   Out\_An\_Red & Sa\'ida & Sinal anal\'ogico para cor vermelha \\
  \midrule
   Out\_Dig\_Red & Sa\'ida & Sinal digital para cor vermelha \\
  \bottomrule
\end{tabular}%
}{%
  \fonte{Produzido pelo autor.}
}
\end{table}

\begin{table}[htb]
\IBGEtab{%
  \caption{Descri{\c c}\~ao dos sinais de entrada e sa\'ida do circuito projetado para os blocos de teste}%
  \label{tab_circcomp2}
}{%
  \begin{tabular}{cccc}
  \toprule
   Sinal & Tipo & Descri{\c c}\~ao & Observa{\c c}\~ao \\
  \bottomrule
\end{tabular}%
}{%
  \fonte{Produzido pelo autor.}
}
\end{table}
%\section{Blocos B\'asicos}
Alguns blocos b\'asicos foram projetados para o desenvolvimento do projeto e ser\~ao relatados ao longo de toda essa se{\c c}\~ao.

\input{Projeto/BlocosBasicos/Inversor}
%\subsection{\NomeBloco}

Um segundo tipo de inversor foi constru\'ido para ser utilizado no bloco Buffer (apresentado na \autoref{secaoBuffer}). Este Inversor apresenta a mesma Tabela Verdade, sinais e circuito do Inversor na \autoref{inversor1}, por\'em com representa{\c c}\~ao em bloco e par\^ametros dos transistores diferentes. A \autoref{fig_inversor2} apresenta a sua representa{\c c}\~ao em bloco, enquanto a \autoref{tab_inversor2} apresenta os par\^ametros de seus componentes.

\begin{figure}[htb]
 \label{fig_inversor2}
 \centering
    \centering
    \caption{Representa{\c c}\~ao em bloco do segundo Inversor } \label{\NomePFig}
    \includegraphics[scale=0.5]{Circuitos/inv_buf_simbolo.png}
    \legend{Fonte: Produzido pelo autor}\end{figure}

Os transistores utilizados no bloco do segundo Inversor apresentam os par\^ametros mostrados na \autoref{\NomeTTab}.

\begin{table}[htbp]
\caption{Sinais do bloco \NomeBloco}
\label{\NomeTTab}
\centering
\begin{tabular}{ccccc}
\toprule
Transistor & W ($\mu$m)  & L ($\mu$m)           & M (n° dispositivos) & S (n° dispositivos)\\
\midrule \midrule
Q1 & 1,2 & 0,18 & 1 & 1\\
\midrule
Q2 & 0,6 & 0,18 & 1 & 1\\
\bottomrule
\end{tabular}
\legend{Fonte: Produzido pelo autor}
\end{table}
\clearpage
\renewcommand{\NomeBloco}{\emph{Porta de Transmissão}}
\renewcommand{\NomeBlocoNoIt}{Porta de Transmissão}
\renewcommand{\NomePTab}{tab_\NomeBlocoNoIt}
\renewcommand{\NomeSTab}{tab_\NomeBlocoNoIt2}
\renewcommand{\NomePFig}{fig_\NomeBlocoNoIt}
\renewcommand{\NomeSFig}{fig_\NomeBlocoNoIt2}
\renewcommand{\NomeTTab}{tab_\NomeBlocoNoIt3}

\subsection{\NomeBloco}

O bloco \emph{\NomeBloco{}} funciona como uma chave, permitindo ou n\~ao o sinal de um lado passar ao outro. O bloco apresenta as defini{\c c}\~oes de sinais de entrada e sa\'ida referidos na \autoref{\NomeSTab}.

\begin{table}[htbp]
\caption{Sinais do bloco \emph{\NomeBloco}}
\label{\NomeSTab}
\centering
\begin{tabular}{ccl}

    \toprule
    Sinal & Tipo    & Descri{\c c}\~ao        \\
    \midrule \midrule
    A & Bidirecional & Sinal bidirecional 1\\
    \midrule
    B & Bidirecional & Sinal bidirecional 2\\
    \midrule
    ENABLE & Entrada & Sinal de habilita{\c c}\~ao\\
    \bottomrule
\end{tabular}
\legend{Fonte: Produzido pelo autor}
\end{table}

O circuito projetado para o bloco \'e demonstrado na \autoref{\NomePFig}.

\begin{figure}[htb]
 \label{\NomePFig}
 \centering
  \begin{minipage}{0.4\textwidth}
    \centering
    \caption{Circuito CMOS projetado para o bloco \emph{\NomeBloco}} \label{\NomePFig}
    \includegraphics[scale=0.5]{Circuitos/TG.png}
    \legend{Fonte: Produzido pelo autor}
  \end{minipage}
  \hfill
  \begin{minipage}{0.4\textwidth}
    \centering
    \caption{Representa{\c c}\~ao em bloco do \emph{\NomeBloco}} \label{\NomeSFig}
    \includegraphics[scale=0.5]{Circuitos/TG_Simbolo.png}
    \legend{Fonte: Produzido pelo autor}
  \end{minipage}
\end{figure}

Os transistores utilizados no bloco \emph{\NomeBloco{}} apresentam os par\^ametros mostrados na \autoref{\NomeTTab}.

\begin{table}[htbp]
\caption{Transistores do Bloco \emph{\NomeBloco}}
\label{\NomeTTab}
\centering
\begin{tabular}{ccccc}
\toprule
Transistor & W ($\mu$m)  & L ($\mu$m)           & M (n° dispositivos) & S (n° dispositivos)\\
\midrule \midrule
Q1 & 0,8 & 0,18 & 1 & 1\\
\midrule
Q2 & 0,4 & 0,18 & 1 & 1\\
\bottomrule
\end{tabular}
\legend{Fonte: Produzido pelo autor}
\end{table}
\clearpage

\renewcommand{\NomeBloco}{Buffer}
\renewcommand{\NomePTab}{tab_\NomeBloco}
\renewcommand{\NomeSTab}{tab_\NomeBloco2}
\renewcommand{\NomePFig}{fig_\NomeBloco}
\renewcommand{\NomeSFig}{fig_\NomeBloco2}
\renewcommand{\NomeTTab}{tab_\NomeBloco3}

\section{Buffer}
\label{buffer}

O bloco \NomeBloco{} tem a finalidade de colocar o mesmo sinal de entrada na sua sa\'ida, reduzindo efeitos de carga. O sinal de entrada pode ser tanto anal\'ogico quanto digital. O bloco apresenta as defini ções de sinais de entrada e sa\'ida referidos na \autoref{\NomeSTab}.

\begin{table}[htb]
\caption{Sinais do bloco \NomeBloco}
\label{\NomeSTab}
\centering
\begin{tabular}{ccl}

    \toprule
    Sinal & Tipo    & Descri ção        \\
    \midrule \midrule
    In    & Entrada & Sinal de Entrada \\
    \midrule
    Out   & Saída   & Sinal de Sa\'ida   \\
    \bottomrule
\end{tabular}
\legend{Fonte: Produzido pelo autor}
\end{table}

O circuito projetado para o bloco \'e demonstrado na \autoref{\NomePFig}.

\begin{figure}[htb]
 \label{NomePFig}
 \centering
    \caption{Circuito CMOS projetado para o bloco \NomeBloco} \label{\NomePFig}
    \includegraphics[scale=0.3]{Circuitos/Buffer.png}
    \legend{Fonte: Produzido pelo autor}
\end{figure}

\begin{figure}[htb]
 \label{NomePFig}
 \centering
    \centering
    \caption{Representa ção em bloco do \NomeBloco} \label{NomeSFig2}
    \includegraphics[scale=0.3]{Circuitos/Buffer_block.png}
    \legend{Fonte: Produzido pelo autor}
\end{figure}


Os transistores utilizados no bloco \NomeBloco{} apresentam os par\^ametros mostrados na \autoref{\NomeTTab}.

\begin{table}[htb]
\caption{Transistores do Bloco \NomeBloco}
\label{\NomeTTab}
\centering
\begin{tabular}{ccccc}
\toprule
Transistor & W ($\mu$m)  & L ($\mu$m)           & M (n° dispositivos) & S (n° dispositivos)\\
\midrule \midrule
Q1 e Q6 & 0,22 & 0,54 & 1 & 1\\
\midrule
Q2 e Q7 & 0,66 & 0,54 & 1 & 1\\
\midrule
Q3 e Q8 & 1,98 & 0,54 & 1 & 1\\
\midrule
Q4 e Q9 & 5,94 & 0,54 & 1 & 1\\
\midrule
Q5 e Q10 & 17,82 & 0,54 & 1 & 1\\
\bottomrule
\end{tabular}
\legend{Fonte: Produzido pelo autor}
\end{table}
\renewcommand{\NomeBloco}{\emph{NAND}}
\renewcommand{\NomeBlocoNoIt}{NAND}
\renewcommand{\NomePTab}{tab_\NomeBlocoNoIt}
\renewcommand{\NomeSTab}{tab_\NomeBlocoNoIt2}
\renewcommand{\NomePFig}{fig_\NomeBlocoNoIt}
\renewcommand{\NomeSFig}{fig_\NomeBlocoNoIt2}
\renewcommand{\NomeTTab}{tab_\NomeBlocoNoIt3}

\section{NAND}

O bloco \NomeBloco{} tem a finalidade de receber duas entradas digitais, e rcolocar o resultado da opera ção NAND em sua sa\'ida. A \autoref{\NomePTab} indica a Tabela Verdade do bloco.

\begin{table}[htbp]

\caption{Tabela Verdade do bloco \NomeBloco}%
\label{\NomePTab}
\centering
\begin{tabular}{ccc}
\toprule
    A & B & Out \\
    \midrule \midrule
    0 & 0 & 1 \\
    \midrule
    0 & 1 & 1\\
    \midrule
    1 & 0 & 1\\
    \midrule
    1 & 1 & 0\\
\bottomrule

\end{tabular}
\fonte{Produzido pelo autor.}
\end{table}

O bloco apresenta as defini ções de sinais de entrada e sa\'ida referidos na \autoref{\NomeSTab}.

\begin{table}[htbp]
\caption{Sinais do bloco \NomeBloco}
\label{\NomeSTab}
\centering
\begin{tabular}{ccl}

    \toprule
    Sinal & Tipo    & Descri ção        \\
    \midrule \midrule
    A    & Entrada & Sinal de Entrada A \\
    \midrule
    B    & Entrada & Sinal de Entrada B \\
    \midrule
    Out    & Sa\'ida & Sinal de sa\'ida \\
    \bottomrule
\end{tabular}
\legend{Fonte: Produzido pelo autor}
\end{table}

O circuito projetado para o bloco \'e demonstrado na \autoref{\NomePFig}.

\begin{figure}[htbp]
 \label{NomePFig}
 \centering
  \begin{minipage}{0.4\textwidth}
    \centering
    \caption{Circuito CMOS projetado para o bloco \NomeBloco} \label{\NomePFig}
    \includegraphics[scale=0.3]{Circuitos/NAND.png}
    \legend{Fonte: Produzido pelo autor}
  \end{minipage}
  \hfill
  \begin{minipage}{0.4\textwidth}
    \centering
    \caption{Representa ção em bloco do \NomeBlocoNoIt} \label{NomeSFig}
    \includegraphics[scale=0.3]{Circuitos/NAND_block.png}
    \legend{Fonte: Produzido pelo autor}
  \end{minipage}
\end{figure}
\renewcommand{\NomeBloco}{\emph{Comparador}}
\renewcommand{\NomeBlocoNoIt}{Comparador}
\renewcommand{\NomePTab}{tab_\NomeBloco}
\renewcommand{\NomeSTab}{tab_\NomeBlocoNoIt2}
\renewcommand{\NomePFig}{fig_\NomeBlocoNoIt}
\renewcommand{\NomeSFig}{fig_\NomeBlocoNoIt2}
\renewcommand{\NomeTTab}{tab_\NomeBlocoNoIt3}

\section{\NomeBloco}

O bloco \NomeBloco{} tem a fun{\c c}\~ao de comparar dois sinais anal\'ogicos advindos nas entradas '+' (\emph{Vp}) e '-' (\emph{Vn}), e retornar \emph{VDD} caso Vp seja maior do que Vn, e \emph{GND} caso Vp seja menor ou igual a Vn.  O bloco apresenta as defini{\c c}\~oes de sinais de entrada e sa\'ida referidos na \autoref{\NomeSTab}.

\begin{table}[htbp]
\caption{Sinais do bloco \NomeBloco}
\label{\NomeSTab}
\centering
\begin{tabular}{ccl}

    \toprule
    Sinal & Tipo    & Descri{\c c}\~ao        \\
    \midrule \midrule
    Vp (+) & Entrada & Entrada positiva do Comparador\\
    \midrule
    Vn (-) & Entrada & Entrada negativa do Comparador\\
    \midrule
    Ibias & Entrada & Corrente de polariza{\c c}\~ao do Comparador\\
    \midrule
    Vo & sa\'ida & Sa\'ida do Comparador\\
    \bottomrule
\end{tabular}
\legend{Fonte: Produzido pelo autor}
\end{table}

O circuito projetado para o bloco \'e demonstrado na \autoref{\NomePFig}.

\begin{figure}[htb]
 \centering
    \centering
    \caption{Circuito CMOS projetado para o bloco \NomeBloco} 
    \includegraphics[scale=0.3]{Circuitos/Comparator.png}
    \legend{Fonte: Produzido pelo autor}
    \label{\NomePFig}
\end{figure}

\begin{figure}[htb]
 \centering
    \centering
    \caption{Representa{\c c}\~ao em bloco do \NomeBloco} \label{\NomeSFig}
    \includegraphics[scale=0.3]{Circuitos/Comparator_block.png}
    \legend{Fonte: Produzido pelo autor}
\end{figure}

Os transistores utilizados no bloco \NomeBloco{} apresentam os par\^ametros mostrados na \autoref{\NomeTTab}.

\begin{table}[htbp]
\caption{Transistores do Bloco \NomeBloco}
\label{\NomeTTab}
\centering
\begin{tabular}{ccccc}
\toprule
Transistor & W ($\mu$m)  & L ($\mu$m)           & M (n° dispositivos) & S (n° dispositivos)\\
\midrule \midrule
Q1 & 10 & 1 & 1 & 1\\
\midrule
Q2$^1$ & 10 & 1 & 6 & 6\\
\midrule
Q3 & 4 & 1 & 2 & 2\\
\midrule
Q4 & 4 & 1 & 2 & 2\\
\midrule
Q5 & 2 & 1 & 2 & 2\\
\midrule
Q6 & 2 & 1 & 2 & 2\\
\midrule
Q7$^1$ & 10 & 1 & 8 & 8\\
\midrule
Q8 & 2 & 1 & 4 & 4\\
\midrule
Q9 & 3 & 0.18 & 1 & 1\\
\midrule
Q10 & 1.5 & 0.18 & 1 & 1\\

\bottomrule
\end{tabular}
\legend{Fonte: Produzido pelo autor}
\legend{$^1$Calculado de forma a produzir uma corrente de 9 $\mu$A}
\end{table}
 
O \NomeBloco{} \'e desenvolvido com tr\^es est\'agios de amplifica{\c c}\~ao. O primeiro est\'agio, composto pelos transistores Q3, Q4, Q5, Q6 t\^em a fun{\c c}\~a de realizar a diferen{\c c}a entra as entradas \emph{Vp} e \emph{Vn} e multiplicar por um pequeno ganho. Os transistores Q3 e Q4 sao respons\'aveis por receber as entradas, enquanto os transistores Q5 e Q6 funcionam como transistores de Carga Ativa. O Q2 funciona como uma fonte de corrente.

O segundo est\'agio \'e um est\'agio de ganho, do qual o transistor Q2 fornece um ganho para a saida do est\'agio anterior e o transistor Q7 funciona como uma fonte de corrente para o est\'agio.

Diodos quadrados \emph{D1} e \emph{D2} de prote{\c c}\~ao s\~ao inclu\'idos nos terminais \emph{Vp} e \emph{Vn do circuito}, com \^anodo ligado ao terra e catodo ligado ao seu respectivo terminal. Os diodo apresentam os seguintes par\^ametros apresentados na \autoref{diodosComp}.

\begin{table}[htbp]
\caption{Diodos do Bloco \NomeBloco}
\label{diodosComp}
\centering
\begin{tabular}{cccc}
\toprule
Diodo & W ($\mu$m)  & L ($\mu$m)           & \'Area ($\mu$m²)\\
\midrule \midrule
D1 e D2 & 1 & 1 & 1 \\

\bottomrule
\end{tabular}
\legend{Fonte: Produzido pelo autor}
\end{table}

\section{Espelhos de Corrente}

Espelhos de Corrente foram necess\'arios para o desenvolvimento de diversos blocos contidos no projeto, e ser\~ao especificados nas subse{\c c}\~oes \` seguinte. Uma explica{\c c}\~ao geral sobre o funcionamento e desenvolvimento de espelhos de corrente pode ser visto no \autoref{anexoespelhos}.

\input{Projeto/Espelhos/ibias_generator}
\renewcommand{\NomeBloco}{iref\_generator}
\renewcommand{\NomeBlocoNoUnderline}{irefgenerator}
\renewcommand{\NomePTab}{tab_\NomeBlocoNoUnderline}
\renewcommand{\NomeSTab}{tab_\NomeBlocoNoUnderline2}
\renewcommand{\NomePFig}{fig_\NomeBlocoNoUnderline}
\renewcommand{\NomeSFig}{fig_\NomeBlocoNoUnderline2}
\renewcommand{\NomeTTab}{tab_\NomeBlocoNoUnderline3}
\renewcommand{\NomeQTab}{tab_\NomeBlocoNoUnderline4}

\subsubsection{\NomeBloco}

O bloco \NomeBloco{} \'e um espelho de corrente que apresenta algumas sa\'idas como fonte e outras em dreno de corrente. O bloco apresenta as defini{\c c}\~oes de sinais de entrada e sa\'ida referidos na \autoref{\NomeSTab}.

\begin{table}[htbp]
\caption{Sinais do bloco \NomeBloco}
\label{\NomeSTab}
\centering
\begin{tabular}{ccl}

    \toprule
    Sinal & Tipo    & Descri{\c c}\~ao        \\
    \midrule \midrule
    iref1\_src   & Saída   & Fonte de Corrente de 0.5 $\mu$A \\
    \midrule
    iref2\_src   & Saída   & Fonte de Corrente de 0.5 $\mu$A \\
    \midrule
    iout\_test   & Saída   & Fonte de Corrente de 1.5 $\mu$A \\
    \midrule
    iref1\_sink   & Saída   & Dreno de Corrente de 0.5 $\mu$A \\
    \midrule
    iref2\_sink   & Saída   & Dreno de Corrente de 0.5 $\mu$A \\
    \bottomrule
\end{tabular}
\legend{Fonte: Produzido pelo autor}
\end{table}

O circuito projetado para o bloco \'e demonstrado na \autoref{\NomePFig}.

\begin{figure}[htb]
 \label{\NomePFig}
 \centering
    \centering
    \caption{Circuito CMOS projetado para o bloco \NomeBloco} 
    \includegraphics[scale=0.3]{Circuitos/iref_generator.png}
    \legend{Fonte: Produzido pelo autor}
    \nota{Nem todos transistores se apresentam representados na imagem. \emph{Q17}, \emph{Q18}, \emph{Q19} e \emph{Q20} s\~ao transistores replicados para cada sa\'ida \emph{IrefX\_src} e \emph{IrefY\_sink}, onde \emph{X} e \emph{Y} s\~ao os identificadores de cada sa\'ida}
\end{figure}

\begin{figure}[htb]
 \centering
    \centering
    \caption{Representa{\c c}\~ao em bloco do \NomeBloco} \label{\NomeSFig}
    \includegraphics[scale=0.5]{Circuitos/iref_generator_block.png}
    \legend{Fonte: Produzido pelo autor}
\end{figure}

Os transistores utilizados no bloco \NomeBloco{} apresentam os par\^ametros mostrados na \autoref{\NomeTTab}.

\begin{table}[htbp]
\caption{Transistores do Bloco \NomeBloco}
\label{\NomeTTab}
\centering
\begin{tabular}{ccccc}
\toprule
Transistor & W ($\mu$m)  & L ($\mu$m)  & M (n° dispositivos) & S (n° dispositivos)\\
\midrule \midrule

\midrule
Q1                                   & 0,3    & 19,995 & 1                   & 3                   \\
\midrule
Q2                                   & 35     & 0,18   & 2                   & 1                   \\
\midrule
Q3 e Q4                              & 15     & 0,18   & 1                   & 1                   \\
\midrule
Q5                                   & 10     & 15     & 1                   & 6                   \\
\midrule
\begin{tabular}[c]{@{}c@{}}Q6, Q9, Q10,\\ Q13, Q19 e Q20\end{tabular}          & 4      & 19,995 & 1                   & 2                   \\
\midrule
\begin{tabular}[c]{@{}c@{}}Q7, Q8, Q11, Q12,\\ Q15, Q17a¹ e Q18a¹\end{tabular} & 10     & 15   & 2                   & 1                   \\
\midrule
Q14                                  & 4      & 19,995 & 10                  & 1                   \\
\midrule
Q16                                  & 4      & 19,995 & 1                   & 6 \\
\midrule
Q17b² e Q18b²                                                                  & 10     & 15     & 6                   & 1                  \\
\bottomrule
\end{tabular}
\legend{Fonte: Produzido pelo autor}
\legend{¹ Q17a e Q18a s\~ao os transistores referentes \`as sa\'idas iref1\_src e iref2\_src\\
² Q17b e Q18b s\~ao os transistores referentes \`a sa\'ida iout\_test}
\end{table}

O resistor \emph{R} utilizado no bloco \NomeBloco{} apresentam os par\^ametros mostrado na \autoref{\NomeQTab}.

\begin{table}[htbp]
\caption{Resistores do bloco \NomeBloco}
\label{\NomeQTab}
\centering
\begin{tabular}{cccc}
\toprule
Resistor & W ($\mu$m)  & L ($\mu$m) & Resist\^encia (k$\Omega$)\\
\midrule \midrule
R & 3 & 404.4 & 141,996\\
\bottomrule
\end{tabular}
\legend{Fonte: Produzido pelo autor}
\end{table}

Os transistores \emph{Q6} e \emph{Q9}, juntos aos resistores \emph{R1} e \emph{R2}, t\^em a finalidade de funcionarem como um dreno de corrente referenciados pelos resistores. Os transistores \emph{Q4} e \emph{Q7} t\^em a finalidade de funcionarem como uma fonte de corrente, referenciados pelo dreno de corrente j\'a mencionado. O transistor \emph{Q10} \'e o bra{\c c}o do espelho de corrente do qual fornecem a corrente de sa\'ida.

Os transistores \emph{Q5}, \emph{Q8} e \emph{Q11} se apresentam na configura{\c c}\~ao \emph{Cascode}, que tem o intuito de tornar o espelho de corrente de resposta mais linear, aumentar sua banda e ainda aumentar as suas resist\^encias de entrada e sa\'ida.

Os transistores \emph{Q1}, \emph{Q2} e \emph{Q3} t\^em a fun{\c c}\~ao de inicializarem o circuito no ponto de opera{\c c}\~ao adequado, j\'a que o circuito tamb\'em apresenta estabilidade quando fornecendo 0 A, sendo necess\'ario evitar essa situa{\c c}\~ao.
\clearpage
\renewcommand{\NomeBloco}{current\_mirror\_nmos}
\renewcommand{\NomeBlocoNoUnderline}{curmirnmosb}
\renewcommand{\NomePTab}{tab_\NomeBlocoNoUnderline}
\renewcommand{\NomeSTab}{tab_\NomeBlocoNoUnderline2}
\renewcommand{\NomePFig}{fig_\NomeBlocoNoUnderline}
\renewcommand{\NomeSFig}{fig_\NomeBlocoNoUnderline2}
\renewcommand{\NomeTTab}{tab_\NomeBlocoNoUnderline3}
\renewcommand{\NomeQTab}{tab_\NomeBlocoNoUnderline4}

\subsubsection{\NomeBloco}

O bloco \NomeBloco{} cont\'em alguns bra{\c c}os utilizados como dreno de corrente para outras partes do circuito, sendo todos valores iguais \'a corrente de refer\^encia. O bloco apresenta as defini{\c c}\~oes de sinais de entrada e sa\'ida referidos na \autoref{\NomeSTab}.

\begin{table}[htbp]
\caption{Sinais do bloco \NomeBloco}
\label{\NomeSTab}
\centering
\begin{tabular}{ccl}

    \toprule
    Sinal & Tipo    & Descri{\c c}\~ao        \\
    \midrule \midrule
    Iref\_bias   & Entrada   &  Corrente de refer\^encia para os bra{\c c}os \\
    \midrule
    Iref\_A   & Saída   &  Bra{\c c}o 1 \\
    \midrule
    Iref\_B   & Saída   &  Bra{\c c}o 2 \\
    \midrule
    Iref\_C   & Saída   &  Bra{\c c}o 3 \\
    \midrule
    Iref\_D   & Saída   &  Bra{\c c}o 4 \\
    \midrule
    Iref\_E   & Saída   &  Bra{\c c}o 5 \\
    \bottomrule
\end{tabular}
\legend{Fonte: Produzido pelo autor}
\end{table}

O circuito projetado para o bloco \'e demonstrado na \autoref{\NomePFig}.

\begin{figure}[htb]
 \centering
    \centering
    \caption{Circuito CMOS projetado para o bloco \NomeBloco} 
    \includegraphics[scale=0.3]{Circuitos/current_mirror.png}
    \legend{Fonte: Produzido pelo autor}
    \label{\NomePFig}
\end{figure}

\begin{figure}[htb]
 \centering
    \centering
    \caption{Representa{\c c}\~ao em bloco do \NomeBloco} \label{\NomeSFig}
    \includegraphics[scale=0.3]{Circuitos/current_mirror_block.png}
    \legend{Fonte: Produzido pelo autor}
\end{figure}

Os transistores utilizados no bloco \NomeBloco{} apresentam os par\^ametros mostrados na \autoref{\NomeTTab}.

\begin{table}[htbp]
\caption{Transistores do Bloco \NomeBloco}
\label{\NomeTTab}
\centering
\begin{tabular}{ccccc}
\toprule
Transistor & W ($\mu$m)  & L ($\mu$m)           & M (n° dispositivos) & S (n° dispositivos)\\
\midrule \midrule
\begin{tabular}[c]{@{}c@{}}Q1, Q2, Q3,\\
Q4, Q5 e Q6\end{tabular} & 5 & 6 & 2 & 1\\
\bottomrule
\end{tabular}
\legend{Fonte: Produzido pelo autor}
\end{table}

\renewcommand{\NomeBloco}{APS\_digitalized}
\renewcommand{\NomeBlocoNoUnderline}{apsdigitalized}
\renewcommand{\NomePTab}{tab_\NomeBlocoNoUnderline}
\renewcommand{\NomeSTab}{tab_\NomeBlocoNoUnderline2}
\renewcommand{\NomePFig}{fig_\NomeBlocoNoUnderline}
\renewcommand{\NomeSFig}{fig_\NomeBlocoNoUnderline2}
\renewcommand{\NomeTTab}{tab_\NomeBlocoNoUnderline3}
\renewcommand{\NomeQTab}{tab_\NomeBlocoNoUnderline4}

\section{\NomeBloco}

O \emph{APS\_digitalized} \'e o circuito respons\'avel por digitalizar o sinal gerado pelo APS descrito na \autoref{section:APS}. O bloco apresenta as defini{\c c}\~oes de sinais de entrada e sa\'ida referidos na \autoref{\NomeSTab}.

\begin{table}[htbp]
\caption{Sinais do bloco \NomeBloco}
\label{\NomeSTab}
\centering
\begin{tabular}{ccll}

    \toprule
    Sinal & Tipo    & Descri{\c c}\~ao & Observa{\c c}\~ao        \\
    \midrule \midrule
    RESET   & Entrada   & Sinal de RESET no APS & Ativo em nível baixo\\
    \midrule
    ENABLE   & Entrada   & Sinal de ENABLE no APS & Ativo em nível alto\\
    \midrule
    Vref   & Entrada   & Tens\~ao de refer\^encia utilizada pelo comparador \\
    \midrule
    Ibias   & Entrada   & Corrente de polariza{\c c}\~ao do comparador \\
    \midrule
    AnOut   & Saída   & Sinal anal\'ogico produzido pelo APS \\
    \midrule
    DigOut   & Saída   & Sinal digital produzido pelo Comparador \\
    \bottomrule
\end{tabular}
\legend{Fonte: Produzido pelo autor}
\end{table}

O circuito projetado para o bloco \'e demonstrado na \autoref{\NomePFig}.

\begin{figure}[htb]
 \label{\NomePFig}
 \centering
    \centering
    \caption{Circuito CMOS projetado para o bloco \NomeBloco} 
    \includegraphics[scale=0.4]{Circuitos/APS_digitalized.png}
    \legend{Fonte: Produzido pelo autor}
\end{figure}

\begin{figure}[htb]
 \centering
    \centering
    \caption{Representa{\c c}\~ao em bloco do \NomeBloco} \label{\NomeSFig}
    \includegraphics[scale=0.5]{Circuitos/APS_digitalized_block.png}
    \legend{Fonte: Produzido pelo autor}
\end{figure}

A sa\'ida digital do bloco funciona realizando uma compara{c c}\~ao entre uma tens\~ao de refer\^encia chamada de \emph{Vref} e a sa\'ida anal\'ogica do APS, chamada de \emph{AnOut}. Quando o valor de \emph{Vref} for maior do que o de \emph{AnOut}, o comparador ir\'a saturar e apresentar em a m\'axima tens\~ao na sa\'ida, que deve ser aproximamente igual a \emph{VDD}, interpretada como n\'ivel l\'ogico '1'. Quando o valor de \emph{Vref} for menor ou igual do que o de \emph{AnOut}, o comparador ir\'a apresentar um sinal de aproximadamente \emph{GND} em sua sa\'ida, interpretada como n\'ivel l\'ogico '0'.
Utilizando o elemento comparador, podemos ajustar para que a sa\'ida retorne '0' apenas quando for atingido um valor limiar controlado. Como sabemos que a intensidade da corrente fotogerada depende da intensidade da luz captada pelo fotodiodo (\autoref{secao_fotodiodo}), podemos deduzir a informa{\c c}\~ao sobre intensidade luminosa verificando em quanto tempo demora para que se mude de n\'ivel l\'ogico '1' para '0' durante o Est\'agio 2, utilizando-se a \autoref{eq_responsividade} e a \autoref{eq_modEletFotIl} e para se obter a \autoref{eq_apsd}.

\begin{equation}
    \label{eq_apsd}
    P = \frac{(VDD-V_{ref})(C_{j}+C_{cn})}{R_{\lambda}T}
\end{equation}

Onde:

\begin{itemize}

    \item \emph{P$_{FD}$} \'e a pot\^encia \'optica presente no fotodiodo [W]
    \item \emph{R$_\lambda$} \'e a Responsividade [A.W$^{-1}$]
    \item \emph{VDD} \'e a tens\~ao de alimenta{\c c}\~ao do circuito [$V$]
    \item \emph{$V_{ref}$} \'e a tens\~ao de refer\^encia do comparador [$V$]
    \item \emph{$C_j$} \'e a capacit\^ancia de jun{\c c}\~ao do fotodiodo [F]
    \item \emph{$C_{cn}$} \'e a capacit\^ancia do n\'o central do APS [F]
    \item $R_{\lambda}$ \'e a responsividade para o comprimento de onda detectado no fotodiodo [$A.W^{-1}$]
    \item \emph{T} \'e o per\'iodo do qual o n\'ivel l\'ogico mudou de '1' para '0' [\emph{s}]
    
\end{itemize}

Um transistor NMOS n\~ao apresentado na ilustra{\c c}\~ao foi conectado com o dreno e fonte ligados ao terra e o gate em VDD, de forma a se formar um capacitor de desacoplamento para o bloco. Os par\^ametros desse transistor s\~ao dadas na \autoref{tab_capdecapsd}.

\begin{table}[htbp]
\caption{Capacitor de desacoplamento via transistor do bloco \NomeBloco}
\label{tab_capdecapsd}
\centering
\begin{tabular}{cccc}
\toprule
W ($\mu$m)  & L ($\mu$m) & M (n° dispositivos) & S (n° dispositivos)\\
\midrule \midrule
10 & 19.995 & 1 & 1\\
\bottomrule
\end{tabular}
\legend{Fonte: Produzido pelo autor}
\end{table}
\clearpage


\renewcommand{\NomeBloco}{APS\_pixel\_clk}
\renewcommand{\NomeBlocoNoUnderline}{apspixelclk}
\renewcommand{\NomePTab}{tab_\NomeBlocoNoUnderline}
\renewcommand{\NomeSTab}{tab_\NomeBlocoNoUnderline2}
\renewcommand{\NomePFig}{fig_\NomeBlocoNoUnderline}
\renewcommand{\NomeSFig}{fig_\NomeBlocoNoUnderline2}
\renewcommand{\NomeTTab}{tab_\NomeBlocoNoUnderline3}
\renewcommand{\NomeQTab}{tab_\NomeBlocoNoUnderline4}

\section{\NomeBloco}

O \NomeBloco{} \'e o bloco respons\'avel por processar e digitalizar o sinal gerado pelo \emph{TIA}. A inten{\c c}\~ao de uso do \emph{TIA} \'e que ele seja o respons\'avel por captar um sinal luminoso com frequ\^encia bem definida, e o sinal el\'etrico gerado, em forma de pulsos quadrados, sirva de refer\^encia de rel\'ogio para todos os circuitos APS utilizados para detec{\c c}\~ao de cor. O bloco portanto tem como responsabilidade gerar um sinal digital de frequ\^encia igual \'a do sinal luminoso captado. O bloco apresenta as defini{\c c}\~oes de sinais de entrada e sa\'ida referidos na \autoref{\NomeSTab}.

\begin{table}[htbp]
\caption{Sinais do bloco \NomeBloco}
\label{\NomeSTab}
\centering
\begin{tabular}{ccl}

    \toprule
    Sinal & Tipo    & Descri{\c c}\~ao\\
    \midrule \midrule
    Vref\_comp   & Entrada   & Tens\~ao de refer\^encia utilizada pelo comparador\\
    \midrule
    Vref\_amp   & Entrada   & Tens\~ao de refer\^encia utilizada para o TIA\\
    \midrule
    Ibias   & Entrada   & Corrente de polariza{\c c}\~ao do comparador \\
    \midrule
    Vout   & Saída   & Sinal digital produzido pelo Comparador\\
    \bottomrule
\end{tabular}
\legend{Fonte: Produzido pelo autor}
\end{table}

O circuito projetado para o bloco \'e demonstrado na \autoref{\NomePFig}.

\begin{figure}[htb]
 \label{\NomePFig}
 \centering
    \centering
    \caption{Circuito CMOS projetado para o bloco \NomeBloco} 
    \includegraphics[scale=0.35]{Circuitos/APS_clk.png}
    \legend{Fonte: Produzido pelo autor}
\end{figure}

\begin{figure}[htb]
 \centering
    \centering
    \caption{Representa{\c c}\~ao em bloco do \NomeBloco} \label{\NomeSFig}
    \includegraphics[scale=0.5]{Circuitos/APS_clk_block.png}
    \legend{Fonte: Produzido pelo autor}
\end{figure}

O circuito tem funcionamento similar ao do apresentado no bloco \emph{APS\_digitalized}, por\'em com apenas a sa\'ida digital sendo considerada e sendo utilizado um $TIA$ ao inv\'es de um APS. A corrente de polariza{\c c}\~ao do par diferencial \'e gerada pelo bloco Ibias\_generator presente no bloco.

Uma resist\^encia $R$ de ganho \'e utilizada, com valor apresentado na \autoref{\NomeQTab}.

\begin{table}[htb]
\caption{Resistores do bloco \NomeBloco}
\label{\NomeQTab}
\centering
\begin{tabular}{cccc}
\toprule
Resistor & W ($\mu$m)  & L ($\mu$m) & Resist\^encia (k$\Omega$)\\
\midrule \midrule
R & 1,68 & 39486,3 & 25000\\
\bottomrule
\end{tabular}
\legend{Fonte: Produzido pelo autor}
\end{table}

Para o bloco, a \autoref{eq_TIAblock} descreve $V_{PH}$, que \'e comparada com \emph{Vref\_comp} e ent\~ao utilizada para gerar o sinal de sa\'ida.

\begin{equation}
    \label{eq_TIAblock}
    V_{PH} = RI_{PH} + Vref_amp
\end{equation}

Onde:

\begin{itemize}

    \item \emph{$V_{PH}$} \'e a tens\~ao no ramo positivo do comparador [$V$]
    \item \emph{$I_{PH}$} \'e a corrente fotogerada [$A$]
    
\end{itemize}

Dois transistores NMOS n\~ao apresentados na ilustra{\c c}\~ao foram conectado com o dreno e fonte ligados ao terra e o gate em VDD, de forma a se formar um capacitor de desacoplamento para o bloco. Os par\^ametros desses transistores s\~ao dadas na \autoref{tab_capdecclk}.

\begin{table}[htbp]
\caption{Capacitores de desacoplamento via transistor do bloco \NomeBloco}
\label{tab_capdecclk}
\centering
\begin{tabular}{cccc}
\toprule
W ($\mu$m)  & L ($\mu$m) & M (n° dispositivos) & S (n° dispositivos)\\
\midrule \midrule
50 & 10 & 1 & 1\\
\bottomrule
\end{tabular}
\legend{Fonte: Produzido pelo autor}
\end{table}

Dois capacitores MOS n\~ao apresentados na ilustra{\c c}\~ao tamb\'em foram utilizados como capacitores de desacoplamento. Os par\^ametros desses capacitores s\~ao dadas na \autoref{tab_capdecclk2}.

\begin{table}[htbp]
\caption{Capacitores de desacoplamento MOS do bloco \NomeBloco}
\label{tab_capdecclk2}
\centering
\begin{tabular}{ccc}
\toprule
W ($\mu$m)  & L ($\mu$m) & Capacit\^ancia ($pF$)\\
\midrule \midrule
30 & 30 & 1,7919\\
\bottomrule
\end{tabular}
\legend{Fonte: Produzido pelo autor}
\end{table}

Dois diodos quadrados ligados reversamente em VDD e GND tamb\'em foi utilizado, com o intuito de proteger o circuito de tens\~oes reversas. Os par\^ametros desses diodos s\~ao dadas na \autoref{tab_capdecclk3}

\begin{table}[htbp]
\caption{Diodos de prote{\c c}\~ao do bloco \NomeBloco}
\label{tab_capdecclk3}
\centering
\begin{tabular}{ccc}
\toprule
W ($\mu$m)  & L ($\mu$m) & \'Area ($um^2$)\\
\midrule \midrule
3 & 3 & 9\\
\bottomrule
\end{tabular}
\legend{Fonte: Produzido pelo autor}
\end{table}
\clearpage
\renewcommand{\NomeBloco}{4\_APS}
\renewcommand{\NomeBlocoNoUnderline}{fouraps}
\renewcommand{\NomePTab}{tab_\NomeBlocoNoUnderline}
\renewcommand{\NomeSTab}{tab_\NomeBlocoNoUnderline2}
\renewcommand{\NomePFig}{fig_\NomeBlocoNoUnderline}
\renewcommand{\NomeSFig}{fig_\NomeBlocoNoUnderline2}
\renewcommand{\NomeTTab}{tab_\NomeBlocoNoUnderline3}
\renewcommand{\NomeQTab}{tab_\NomeBlocoNoUnderline4}

\section{\NomeBloco}

O \emph{\NomeBloco} \'e o circuito respons\'avel por armazenar todos os tr\^es blocos APS de cor (Azul, Verde, Vermelho), mais o TIA geradora de rel\'ogio de refer\^encia para os APS's citados. O bloco apresenta as defini{\c c}\~oes de sinais de entrada e sa\'ida referidos na \autoref{\NomeSTab}.

\begin{table}[htb]
\centering
\IBGEtab{%
  \caption{Descri{\c c}\~ao dos sinais de entrada e sa\'ida do circuito projetado para as cores azul, verde e vermelha}%
  \label{\NomeSTab}
}{%
  \begin{tabular}{ccll}
  \toprule
   Sinal & Tipo & Descri{\c c}\~ao & Observa{\c c}\~ao \\
  \midrule \midrule
   RESET\_BLUE & Entrada & \begin{tabular}[c]{@{}c@{}}Sinal de \emph{RESET} no APS\\ para cor azul\end{tabular} & Ativo em n\'ivel baixo \\
  \midrule
   RESET\_GREEN & Entrada & \begin{tabular}[c]{@{}c@{}}Sinal de \emph{RESET} no APS\\ para cor verde\end{tabular} & Ativo em n\'ivel baixo \\
  \midrule
   RESET\_RED & Entrada & \begin{tabular}[c]{@{}c@{}}Sinal de \emph{RESET} no APS\\ para cor vermelha\end{tabular} & Ativo em n\'ivel baixo \\
   \midrule
   ENABLE\_BLUE & Entrada & \begin{tabular}[c]{@{}c@{}}Sinal de \emph{ENABLE} no APS\\ para cor azul\end{tabular} & Ativo em n\'ivel alto \\
  \midrule
   ENABLE\_GREEN & Entrada & \begin{tabular}[c]{@{}c@{}}Sinal de \emph{ENABLE} no APS\\ para cor verde\end{tabular} & Ativo em n\'ivel alto \\
  \midrule
   ENABLE\_RED & Entrada & \begin{tabular}[c]{@{}c@{}}Sinal de \emph{ENABLE} no APS\\ para cor vermelha\end{tabular} & Ativo em n\'ivel alto \\
  \midrule
   Ibias\_comp1 & Entrada & Fonte de corrente para cor Azul &  \\
   \midrule
   Ibias\_comp2 & Entrada & Fonte de corrente para cor Verde &  \\
   \midrule
   Ibias\_comp3 & Entrada & Fonte de corrente para cor Vermelha &  \\
   \midrule
   Ibias\_clk & Entrada & \begin{tabular}[c]{@{}c@{}}Fonte de corrente para o bloco\\ \emph{APS\_pixel\_clk}\end{tabular}
    &  \\
  \midrule
   Out\_An\_Blue & Sa\'ida & Sinal anal\'ogico para cor azul \\
  \midrule
   Out\_Dig\_Blue & Sa\'ida & Sinal digital para cor azul \\
  \midrule
   Out\_An\_Green & Sa\'ida & Sinal anal\'ogico para cor verde \\
  \midrule
   Out\_Dig\_Green & Sa\'ida & Sinal digital para cor verde \\
  \midrule
   Out\_An\_Red & Sa\'ida & Sinal anal\'ogico para cor vermelha \\
  \midrule
   Out\_Dig\_Red & Sa\'ida & Sinal digital para cor vermelha \\
  \bottomrule
\end{tabular}%
}{%
  \fonte{Produzido pelo autor.}
}
\end{table}

O circuito projetado para o bloco \'e demonstrado na \autoref{\NomePFig}.

\begin{figure}[htb]
 \label{\NomePFig}
 \centering
    \centering
    \caption{Circuito CMOS projetado para o bloco \NomeBloco} 
    \includegraphics[scale=0.3]{Circuitos/4APS.png}
    \legend{Fonte: Produzido pelo autor}
\end{figure}

\begin{figure}[htb]
 \centering
    \centering
    \caption{Representa{\c c}\~ao em bloco do \NomeBloco} \label{\NomeSFig}
    \includegraphics[scale=0.3]{Circuitos/4APS_block.png}
    \legend{Fonte: Produzido pelo autor}
\end{figure}

As sa\'idas digitais de cada bloco APS\_digitalized apresentam um buffer, de forma a garantir a integridade do sinal nos pinos do Circuito Integrado.
\clearpage
\renewcommand{\NomeBloco}{Vref\_generator}
\renewcommand{\NomeBlocoNoUnderline}{vrefgenerator}
\renewcommand{\NomePTab}{tab_\NomeBlocoNoUnderline}
\renewcommand{\NomeSTab}{tab_\NomeBlocoNoUnderline2}
\renewcommand{\NomePFig}{fig_\NomeBlocoNoUnderline}
\renewcommand{\NomeSFig}{fig_\NomeBlocoNoUnderline2}
\renewcommand{\NomeTTab}{tab_\NomeBlocoNoUnderline3}
\renewcommand{\NomeQTab}{tab_\NomeBlocoNoUnderline4}

\section{\NomeBloco}

O \emph{APS\_digitalized} \'e o bloco respons\'avel por gerar todas as tens\~oes de refer\^encia utilizados nos outros circuitos. O bloco apresenta as defini{\c c}\~oes de sinais de entrada e sa\'ida referidos na \autoref{\NomeSTab}.

\begin{table}[htbp]
\caption{Sinais do bloco \NomeBloco}
\label{\NomeSTab}
\centering
\begin{tabular}{ccll}

    \toprule
    Sinal & Tipo    & Descri{\c c}\~ao & Observa{\c c}\~ao        \\
    \midrule \midrule
    Ibias   & Entrada   & Corrente de polariza{\c c}\~ao do bloco Par Diferencial \\
    \midrule
    Vref\_extra   & Saída   & Tens\~ao de refer\^encia 1 \\
    \midrule
    Vref\_plus   & Saída   & Tens\~ao de refer\^encia 2 \\
    \midrule
    Vref   & Saída   & Tens\~ao de refer\^encia 3 \\
    \midrule
    Vref\_minus   & Saída   & Tens\~ao de refer\^encia 4 \\
    \midrule
    Vref\_minus2   & Saída   & Tens\~ao de refer\^encia 5 \\
    \midrule
    Vref\_minus3   & Saída   & Tens\~ao de refer\^encia 6 \\
    \midrule
    Vref\_minus4  & Saída   & Tens\~ao de refer\^encia 7 \\
    \midrule
    Vref\_minus5   & Saída   & Tens\~ao de refer\^encia 8 \\
    \midrule
    Vref\_minus6   & Saída   & Tens\~ao de refer\^encia 9 \\
    \bottomrule
\end{tabular}
\legend{Fonte: Produzido pelo autor}
\end{table}

O circuito projetado para o bloco \'e demonstrado na \autoref{\NomePFig}.

\begin{figure}[htb]
 \label{\NomePFig}
 \centering
    \centering
    \caption{Circuito CMOS projetado para o bloco \NomeBloco} 
    \includegraphics[scale=0.3]{Circuitos/vref_generator.png}
    \legend{Fonte: Produzido pelo autor}
\end{figure}

\begin{figure}[htb]
 \centering
    \centering
    \caption{Representa{\c c}\~ao em bloco do \NomeBloco} \label{\NomeSFig}
    \includegraphics[scale=0.3]{Circuitos/vref_generator_block.png}
    \legend{Fonte: Produzido pelo autor}
\end{figure}

A sa\'ida digital do bloco funciona realizando uma compara{c c}\~ao entre uma tens\~ao de refer\^encia chamada de \emph{Vref} e a sa\'ida anal\'ogica do APS, chamada de \emph{AnOut}. Quando o valor de \emph{Vref} for maior do que o de \emph{AnOut}, o comparador ir\'a saturar e apresentar em a m\'axima tens\~ao na sa\'ida, que deve ser aproximamente igual a \emph{VDD}, interpretada como n\'ivel l\'ogico '1'. Quando o valor de \emph{Vref} for menor ou igual do que o de \emph{AnOut}, o comparador ir\'a apresentar um sinal de aproximadamente \emph{GND} em sua sa\'ida, interpretada como n\'ivel l\'ogico '0'.
Utilizando o elemento comparador, podemos ajustar para que a sa\'ida retorne '0' apenas quando for atingido um valor limiar controlado. Como sabemos que a intensidade da corrente fotogerada depende da intensidade da luz captada pelo fotodiodo (\autoref{secao_fotodiodo}), podemos deduzir a informa{\c c}\~ao sobre intensidade luminosa verificando em quanto tempo demora para que se mude de n\'ivel l\'ogico '1' para '0' durante o Est\'agio 2, utilizando-se a \autoref{eq_responsividade} e a \autoref{eq_modEletFotIl} e para se obter a \autoref{eq_apsd}.

\begin{equation}
    \label{eq_apsd}
    P = \frac{(VDD-V_{ref})(C_{j}+C_{cn})}{R_{\lambda}T}
\end{equation}

Onde:

\begin{itemize}

    \item \emph{P$_{FD}$} \'e a pot\^encia \'optica presente no fotodiodo [W]
    \item \emph{R$_\lambda$} \'e a Responsividade [A.W$^{-1}$]
    \item \emph{VDD} \'e a tens\~ao de alimenta{\c c}\~ao do circuito [$V$]
    \item \emph{$V_{ref}$} \'e a tens\~ao de refer\^encia do comparador [$V$]
    \item \emph{$C_j$} \'e a capacit\^ancia de jun{\c c}\~ao do fotodiodo [F]
    \item \emph{$C_{cn}$} \'e a capacit\^ancia do n\'o central do APS [F]
    \item $R_{\lambda}$ \'e a responsividade para o comprimento de onda detectado no fotodiodo [$A.W^{-1}$]
    \item \emph{T} \'e o per\'iodo do qual o n\'ivel l\'ogico mudou de '1' para '0' [\emph{s}]
    
\end{itemize}

Um transistor NMOS n\~ao apresentado na ilustra{\c c}\~ao foi conectado com o dreno e fonte ligados ao terra e o gate em VDD, de forma a se formar um capacitor de desacoplamento para o bloco. Os par\^ametros desse transistor s\~ao dadas na \autoref{tab_capdecapsd}.

\begin{table}[htbp]
\caption{Capacitor de desacoplamento via transistor do bloco \NomeBloco}
\label{tab_capdecapsd}
\centering
\begin{tabular}{cccc}
\toprule
W ($\mu$m)  & L ($\mu$m) & M (n° dispositivos) & S (n° dispositivos)\\
\midrule \midrule
10 & 19.995 & 1 & 1\\
\bottomrule
\end{tabular}
\legend{Fonte: Produzido pelo autor}
\end{table}
\clearpage


\renewcommand{\NomeBloco}{vref\_block}
\renewcommand{\NomePTab}{tab_\NomeBloco}
\renewcommand{\NomeSTab}{tab_\NomeBloco2}
\renewcommand{\NomePFig}{fig_\NomeBloco}
\renewcommand{\NomeSFig}{fig_\NomeBloco2}
\renewcommand{\NomeTTab}{tab_\NomeBloco3}

O bloco \NomeBloco{} tem a finalidade de conter o bloco \emph{vref\_generator}, mais o bloco respons\'avel por gerar a corrente que o polariza. O bloco apresenta as defini{\c c}\~oes de sinais de entrada e sa\'ida referidos na \autoref{\NomeSTab}.

\begin{table}[htbp]
\caption{Sinais do bloco \NomeBloco}
\label{\NomeSTab}
\centering
\begin{tabular}{ccl}

    \toprule
    Sinal & Tipo    & Descri{\c c}\~ao      \\
    \midrule \midrule
    V\_extra   & Saída   & Tens\~ao de refer\^encia 1 \\
    \midrule
    Vref\_plus   & Saída   & Tens\~ao de refer\^encia 2 \\
    \midrule
    Vref   & Saída   & Tens\~ao de refer\^encia 3 \\
    \midrule
    Vref\_minus   & Saída   & Tens\~ao de refer\^encia 4 \\
    \midrule
    Vref\_minus2   & Saída   & Tens\~ao de refer\^encia 5 \\
    \midrule
    Vref\_minus3   & Saída   & Tens\~ao de refer\^encia 6 \\
    \midrule
    Vref\_minus4  & Saída   & Tens\~ao de refer\^encia 7 \\
    \midrule
    Vref\_minus5   & Saída   & Tens\~ao de refer\^encia 8 \\
    \midrule
    Vref\_minus6   & Saída   & Tens\~ao de refer\^encia 9 \\
    \bottomrule
\end{tabular}
\legend{Fonte: Produzido pelo autor}
\end{table}

O circuito projetado para o bloco \'e demonstrado na \autoref{\NomePFig}.

\begin{figure}[htb]
 \label{NomePFig}
 \centering
    \centering
    \caption{Circuito CMOS projetado para o bloco \NomeBloco} \label{\NomePFig}
    \includegraphics[scale=0.4]{Circuitos/vref_block.png}
    \legend{Fonte: Produzido pelo autor}
\end{figure}

\begin{figure}[htb]
 \label{NomeSFig}
 \centering
    \centering
    \caption{Representa{\c c}\~ao em bloco do \NomeBloco} \label{NomeSFig}
    \includegraphics[scale=0.5]{Circuitos/vref_block_block.png}
    \legend{Fonte: Produzido pelo autor}
\end{figure}

Os transistores utilizados no bloco \NomeBloco{} apresentam os par\^ametros mostrados na \autoref{\NomeTTab}.

\begin{table}[htbp]
\caption{Transistores do Bloco \NomeBloco}
\label{\NomeTTab}
\centering
\begin{tabular}{ccccc}
\toprule
Transistor & W ($\mu$m)  & L ($\mu$m)           & M (n° dispositivos) & S (n° dispositivos)\\
\midrule \midrule
Q1 & 1,2 & 0,18 & 1 & 1\\
\midrule
Q2 & 0,6 & 0,18 & 1 & 1\\
\bottomrule
\end{tabular}
\legend{Fonte: Produzido pelo autor}
\end{table}
\clearpage
\renewcommand{\NomeBloco}{vref\_block\_with\_mux}
\renewcommand{\NomeBlocoA}{vrefblockwithmux}
\renewcommand{\NomePTab}{tab_\NomeBlocoA}
\renewcommand{\NomeSTab}{tab_\NomeBlocoA2}
\renewcommand{\NomePFig}{fig_\NomeBlocoA}
\renewcommand{\NomeSFig}{fig_\NomeBlocoA2}
\renewcommand{\NomeTTab}{tab_\NomeBlocoA3}

O bloco \NomeBloco{} tem a finalidade de selecionar a tens\~ao de refer\^encia a ser utilizada nos blocos de compara{\c c}\~ao do circuito, al\'em de retornar a tens\~ao de refer\^encia utilizada pelo bloco \emph{TIA}. A \autoref{\NomePTab} indica a Tabela Verdade do bloco. Embora tenha uma l\'ogica digital, o circuito permite sa\'idas anal\'ogicas.

\begin{table}[htbp]

\caption{Tabela Verdade do bloco \NomeBloco}%
\label{\NomePTab}
\centering
\begin{tabular}{ccccc}
    \toprule
    D0 & D1 & D2 & D3 & Out \\
    \midrule \midrule
    0 & 0 & 0 & 0 & Vref\_plus\\
    \midrule
    0 & 1 & 0 & 0 & Vref\\
    \midrule
    1 & 0 & 0 & 0 & Vref\_minus2\\
    \midrule
    1 & 1 & 0 & 0 & Vref\_minus3\\
    \midrule
    X & X & 0 & 1 & Vref\_minus4\\
    \midrule
    X & X & 1 & 0 & Vref\_minus5\\
    \midrule
    X & X & 1 & 1 & Vref\_minus6\\
\bottomrule

\end{tabular}
\fonte{Produzido pelo autor.}
\end{table}

O bloco apresenta as defini{\c c}\~oes de sinais de entrada e sa\'ida referidos na \autoref{\NomeSTab}.

\begin{table}[htbp]
\caption{Sinais do bloco \NomeBloco}
\label{\NomeSTab}
\centering
\begin{tabular}{ccl}

    \toprule
    Sinal & Tipo    & Descri{\c c}\~ao      \\
    \midrule \midrule
    D0   & Entrada   & Entrada de sele{\c c}\~ao 1 \\
    \midrule
    D1   & Entrada   & Entrada de sele{\c c}\~ao 2 \\
    \midrule
    D2   & Entrada   & Entrada de sele{\c c}\~ao 3 \\
    \midrule
    D3   & Entrada   & Entrada de sele{\c c}\~ao 4 \\
    \midrule
    Vref\_pixel   & Sa\'ida   & Tens\~ao de refer\^encia selecionada \\
    \midrule
    Vref\_clk¹  & Sa\'ida   & Tens\~ao de refer\^encia do Clock \\
    \midrule
    Vref\_comp²  & Sa\'ida   & Tens\~ao de refer\^encia do Clock do bloco teste \\
    \bottomrule
\end{tabular}
\legend{Fonte: Produzido pelo autor}
\legend{¹ Essa tens\~ao \'e igual \'a sa\'ida Vref\_minus do bloco \emph{vref\_block}\\² Essa tens\~ao \'e igual \'a sa\'ida Vref\_extra do bloco \emph{vref\_block}}
\end{table}

O circuito projetado para o bloco \'e demonstrado na \autoref{\NomePFig}.

\begin{figure}[htb]
 \label{NomePFig}
 \centering
    \centering
    \caption{Circuito CMOS projetado para o bloco \NomeBloco} \label{\NomePFig}
    \includegraphics[scale=0.28]{Circuitos/vref_block.png}
    \legend{Fonte: Produzido pelo autor}
\end{figure}

\begin{figure}[htb]
 \label{NomeSFig}
 \centering
    \centering
    \caption{Representa{\c c}\~ao em bloco do \NomeBloco} \label{NomeSFig}
    \includegraphics[scale=0.4]{Circuitos/vref_block_block.png}
    \legend{Fonte: Produzido pelo autor}
\end{figure}
\clearpage
\input{Projeto/ibias_block}
\renewcommand{\NomeBloco}{\emph{Comparador}}
\renewcommand{\NomeBlocoNoIt}{Comparador}
\renewcommand{\NomePTab}{tab_\NomeBloco}
\renewcommand{\NomeSTab}{tab_\NomeBlocoNoIt2}
\renewcommand{\NomePFig}{fig_\NomeBlocoNoIt}
\renewcommand{\NomeSFig}{fig_\NomeBlocoNoIt2}
\renewcommand{\NomeTTab}{tab_\NomeBlocoNoIt3}

\section{\NomeBloco}

O bloco \NomeBloco{} tem a fun{\c c}\~ao de comparar dois sinais anal\'ogicos advindos nas entradas '+' (\emph{Vp}) e '-' (\emph{Vn}), e retornar \emph{VDD} caso Vp seja maior do que Vn, e \emph{GND} caso Vp seja menor ou igual a Vn.  O bloco apresenta as defini{\c c}\~oes de sinais de entrada e sa\'ida referidos na \autoref{\NomeSTab}.

\begin{table}[htbp]
\caption{Sinais do bloco \NomeBloco}
\label{\NomeSTab}
\centering
\begin{tabular}{ccl}

    \toprule
    Sinal & Tipo    & Descri{\c c}\~ao        \\
    \midrule \midrule
    Vp (+) & Entrada & Entrada positiva do Comparador\\
    \midrule
    Vn (-) & Entrada & Entrada negativa do Comparador\\
    \midrule
    Ibias & Entrada & Corrente de polariza{\c c}\~ao do Comparador\\
    \midrule
    Vo & sa\'ida & Sa\'ida do Comparador\\
    \bottomrule
\end{tabular}
\legend{Fonte: Produzido pelo autor}
\end{table}

O circuito projetado para o bloco \'e demonstrado na \autoref{\NomePFig}.

\begin{figure}[htb]
 \centering
    \centering
    \caption{Circuito CMOS projetado para o bloco \NomeBloco} 
    \includegraphics[scale=0.3]{Circuitos/Comparator.png}
    \legend{Fonte: Produzido pelo autor}
    \label{\NomePFig}
\end{figure}

\begin{figure}[htb]
 \centering
    \centering
    \caption{Representa{\c c}\~ao em bloco do \NomeBloco} \label{\NomeSFig}
    \includegraphics[scale=0.3]{Circuitos/Comparator_block.png}
    \legend{Fonte: Produzido pelo autor}
\end{figure}

Os transistores utilizados no bloco \NomeBloco{} apresentam os par\^ametros mostrados na \autoref{\NomeTTab}.

\begin{table}[htbp]
\caption{Transistores do Bloco \NomeBloco}
\label{\NomeTTab}
\centering
\begin{tabular}{ccccc}
\toprule
Transistor & W ($\mu$m)  & L ($\mu$m)           & M (n° dispositivos) & S (n° dispositivos)\\
\midrule \midrule
Q1 & 10 & 1 & 1 & 1\\
\midrule
Q2$^1$ & 10 & 1 & 6 & 6\\
\midrule
Q3 & 4 & 1 & 2 & 2\\
\midrule
Q4 & 4 & 1 & 2 & 2\\
\midrule
Q5 & 2 & 1 & 2 & 2\\
\midrule
Q6 & 2 & 1 & 2 & 2\\
\midrule
Q7$^1$ & 10 & 1 & 8 & 8\\
\midrule
Q8 & 2 & 1 & 4 & 4\\
\midrule
Q9 & 3 & 0.18 & 1 & 1\\
\midrule
Q10 & 1.5 & 0.18 & 1 & 1\\

\bottomrule
\end{tabular}
\legend{Fonte: Produzido pelo autor}
\legend{$^1$Calculado de forma a produzir uma corrente de 9 $\mu$A}
\end{table}
 
O \NomeBloco{} \'e desenvolvido com tr\^es est\'agios de amplifica{\c c}\~ao. O primeiro est\'agio, composto pelos transistores Q3, Q4, Q5, Q6 t\^em a fun{\c c}\~a de realizar a diferen{\c c}a entra as entradas \emph{Vp} e \emph{Vn} e multiplicar por um pequeno ganho. Os transistores Q3 e Q4 sao respons\'aveis por receber as entradas, enquanto os transistores Q5 e Q6 funcionam como transistores de Carga Ativa. O Q2 funciona como uma fonte de corrente.

O segundo est\'agio \'e um est\'agio de ganho, do qual o transistor Q2 fornece um ganho para a saida do est\'agio anterior e o transistor Q7 funciona como uma fonte de corrente para o est\'agio.

Diodos quadrados \emph{D1} e \emph{D2} de prote{\c c}\~ao s\~ao inclu\'idos nos terminais \emph{Vp} e \emph{Vn do circuito}, com \^anodo ligado ao terra e catodo ligado ao seu respectivo terminal. Os diodo apresentam os seguintes par\^ametros apresentados na \autoref{diodosComp}.

\begin{table}[htbp]
\caption{Diodos do Bloco \NomeBloco}
\label{diodosComp}
\centering
\begin{tabular}{cccc}
\toprule
Diodo & W ($\mu$m)  & L ($\mu$m)           & \'Area ($\mu$m²)\\
\midrule \midrule
D1 e D2 & 1 & 1 & 1 \\

\bottomrule
\end{tabular}
\legend{Fonte: Produzido pelo autor}
\end{table}

\chapter{Resultados}
\section{Simulações}

Simulações foram realizadas para testar os dispositivos projetados. Os testes avaliaram o funcionamento geral do circuito de acordo com as especificações do projeto, e também o funcionamento dos componentes APS e TIA.

Para a realização das simulações, é necessário se realizar uma estimativa da fotocorrente gerada dos fotodiodos. Como até o momento da simulação não havia a disponibilidade de uma amostra do fotodiodo, o autor fez uma estimativa conforme o trabalho de \cite{LidianeCampos}, apresentado na \autoref{tab_estcur}.

\begin{table}[!h]
\caption{Estimativa de faixa de fotocorrente gerada}
\label{tab_estcur}
\centering
\begin{tabular}{cc}
\toprule
& Corrente nA \\
\midrule \midrule
Mínimo & 0,1\\
\midrule
Máximo & 30\\
\bottomrule
\end{tabular}
\legend{Fonte: Produzido pelo autor baseado no trabalho \cite{LidianeCampos}}
\end{table}

\subsection{Máximo sinal DC do APS}
\label{DCAPS}

É avaliado qual é o sinal DC que o bloco APS apresenta sem ter nenhuma fotocorrente gerada, $T_{enable}$ desabilitado e $T_{reset}$ habilitado. Na situação $V_{cn}$ irá apresentar o valor de VDD, o que vai representar o máximo sinal possível em $V_{out}$. A corrente \textit{Iref} considerada foi de $500$ nA, fornecida pelo bloco \textit{current\_mirror\_nmos}. Uma carga de $100$ fF foi utilizada na saída do APS de forma a simular a conexão com  um transistor na saída. O valor obtido foi de $1,15813$ V para $V_{out}$, valor que representa está dentro de VDD menos V\textsubscript{GS} do $T_{buffer}$.

\subsection{Análise transiente do sinal de saída do APS}

É avaliado qual a resposta do APS com a presença de uma fotocorrente. Para a simulação, foi considerado o modelo apresentado na \autoref{fig_APS_cap}, do qual há uma fonte de corrente em paralelo a um diodo representando a fotogeração. A corrente \textit{Iref} considerada foi de $500$ nA, fornecida pelo bloco \textit{current\_mirror\_nmos}.

O Período de Reset nas simulações é definido para ter uma duração de $1$ us. Dois cenários de simulação são descritos nas subseções \ref{sub_sec121} e \ref{sub_sec247} são apresentados, sendo suficientes para comprovação da operação do circuito. O primeiro cenário modela o sinal de fotocorrente gerada com uma onda quadrada de 8 kHz (período de 125 $\mu$s) e valor de amplitude de 5 nA. O segundo cenário modela o sinal de fotocorrente gerada com uma onda quadrada de 4 kHz (período de 250 $\mu$s) e valor de amplitude de 2,5 nA. Esses e outros valores de corrente fotogeradas sugeridos pelo trabalho \cite{LidianeCampos} foram testados na topologia implementada.

\subsection{Período de Integração igual a 121 $\mu$s}
\label{sub_sec121}

Nessa situação, foi considerado uma fotocorrente de 5 nA, que está dentro da faixa proposta pelo autor na \autoref{tab_estcur}. Os resultados obtidos são observados nos gráficos abaixo.

As figuras \ref{graf125} e \ref{graf1252} apresentam o comportamento esperado. Assim que RESET é desabilitado (vai para nível lógico '1'), Vout\_analogico passa a decair linearmente, como mostra \autoref{eq_modEletFotIl}\footnote{Note que aqui a tensão não foi completamente carregada para VDD, e por isso, ela inicia o ciclo de descarregamento em um ponto diferente de VDD, o que deve ser levado em conta na utilização da \autoref{eq_voutfinal}}. Quando Vout\_analogico atinge o valor de Vref, Vout\_digital vai do nível lógico '0' para '1', como descrito na \autoref{sec_apsdigitalized}.

\begin{figure}[!h]
 \centering
    \centering
    \caption{Tensão de saída analógica junto aos sinais de RESET e ENABLE para Período de Integração igual a 121 $\mu$s}
    \includegraphics[scale=0.6]{Resultados/Graficos/reseteenable-tb_pixel125.png}
    \legend{Fonte: Produzido pelo autor}
    \legend{Nota: Vout\_analogico representa a tensão de saída analógica do APS}
    \label{graf125}
\end{figure}

\begin{figure}[!h]
 \centering
    \caption{Tensão de saída analógica e tensão de saída digital comparado à tensão de referência do Comparador Período de Integração igual a 121 $\mu$s} 
    \includegraphics[scale=0.6]{Resultados/Graficos/analogicoedigital-tb_pixel125.png}
    \legend{Fonte: Produzido pelo autor}
    \legend{Nota: Vout\_digital representa a tensão de saída digital do APS\_digitalized}
    \label{graf1252}
\end{figure}

\subsection{Período de Integração igual a 247 $\mu$s}
\label{sub_sec247}

Nessa situação, foi considerado uma fotocorrente de 2.5 nA, que está dentro da faixa proposta pelo autor na \autoref{tab_estcur}. Os resultados obtidos são observados nos gráficos abaixo.

As figuras \ref{graf247} e \ref{graf2472} apresentam mesmo comportamento daqueles apresentados na \autoref{sub_sec121}, como esperado.

\begin{figure}[!h]
 \centering
    \centering
    \caption{Tensão de saída analógica junto aos sinais de RESET e ENABLE para Período de Integração igual a 247 $\mu$s}
    \includegraphics[scale=0.6]{Resultados/Graficos/reseteenable-tb_pixel250.png}
    \legend{Fonte: Produzido pelo autor}
    \legend{Nota: Vout\_analogico representa a tensão de saída analógica do APS}
    \label{graf247}
\end{figure}

\begin{figure}[!h]
 \centering
    \caption{Tensão de saída analógica e tensão de saída digital comparado à tensão de referência do Comparador Período de Integração igual a 247 $\mu$s} 
    \includegraphics[scale=0.6]{Resultados/Graficos/analogicoedigital-tb_pixel250.png}
    \legend{Fonte: Produzido pelo autor}
    \legend{Nota: Vout\_digital representa a tensão de saída digital do APS\_digitalized}
    \label{graf2472}
\end{figure}

\subsection{Avaliação de ruído do APS\_digitalized}

Uma simulação para verificar a resposta ao ruído do APS\_digitalized foi realizado, de forma a verificar o impacto que um ruído pode ter ao sistema.

\begin{figure}[!h]
 \centering
    \caption{Iterações de ruídos na faixa de 1 mV realizadas para simulação} 
    \includegraphics[scale=0.6]{Resultados/Graficos/ruido-Vout_noise.png}
    \legend{Fonte: Produzido pelo autor}
    \label{grafruido}
\end{figure}

A \autoref{grafruido} apresenta um gráfico com todas as iterações geradas para simulação, com sinal de tensão de ruído aplicado à saída do APS, que apresenta $T_{enable}$ desabilitado e $T_{reset}$ habilitado. Nessa situação, o autor do trabalho selecionou aleatoriamente 5 amostras do qual são demonstradas na , mas que são suficientes para as conclusões oferecidas.

\begin{figure}[!h]
 \centering
    \caption{Simulações de gráfico de ruído} 
    \includegraphics[scale=0.8]{Resultados/Graficos/tb_pixel_TRAN_NOISE.png}
    \legend{Fonte: Produzido pelo autor}
    \label{grafruido2}
\end{figure}

Como pode ser observado pela \autoref{grafruido2}, o ruído não impactou de forma visível a resposta do circuito, aparentando estarem todos visualmente sobre a mesma linha do gráfico.

\subsection{Análise transiente do sinal de saída do TIA}

Uma simulação do TIA foi realizada para averiguar se seu comportamento se apresenta adequado. Para as simulações foi utilizado Vref\_amp igual a 800 mV, de forma a polarizar reversamente o fotodiodo e consequentemente aumenta a sua região de depleção, além de sua linearidade; e Vref\_comp igual a 1.3 V utilizado como tensão de referência de comparação no processo de digitalização do comparador.

\begin{figure}[!h]
 \centering
    \caption{Tensão de saída analógica do TIA} 
    \includegraphics[scale=0.7]{Resultados/Graficos/tb_clock.png}
    \legend{Fonte: Produzido pelo autor}
    \label{graf_tiasinal}
\end{figure}

O comportamento dos gráficos mostra um resultado adequado. Para uma fotocorrente de sinal quadrado, uma tensão quadrada de mesmo formato é gerada para a saída analógica e a digital.


\section{Layout}

Com os resultados de simulação validando o funcionamento do circuito, foi possível iniciar o desenvolvimento do layout. Todo o projeto foi realizado utilizando a ferramenta \textit{Virtuoso}, da \textit{Cadence}, utilizando o processo \textit{TSMC CMOS 180 nm}. O circuito integrado apresentado na \autoref{fig_circintegrado} foi desenvolvido, contendo todo o projeto de Receptor Óptico além de outros projetos desenvolvidos terceiros\footnote{"Projeto de um Oscilador em Anel Controlado por Tensão de Múltiplas Saídas em Tecnologia CMOS 180 nm" \cite{VictorRodrigues}}\footnote{"QUAL OS OUTROS ???????????????????????}. A \autoref{fig_circintegrado_division} mostra de maneira explicita o que representa cada parte do CI. O circuito integrado apresenta uma área total de 2x2 mm² (2,56 mm²), utilizando um encapsulamento do tipo CLCC44\footnote{Especificações do encapsulamento presentes no \autoref{anexo_clcc44}}. A \autoref{tab_clcc44} apresenta a relação da numeração dos pinos do encapsulamento com os pinos chip, que são distintos, além da identificação de cada pino. A \autoref{tab_clcc44_2} apresenta outros pinos presentes no encapsulamento, referente à outros trabalhos.

\begin{figure}[!h]
 \centering
    \caption{Circuito Integrado utilizado para o Receptor Óptico} 
    \includegraphics[scale=0.5]{Resultados/Imagens/CircuitoIntegrado.png}
    \legend{Fonte: Produzido pelo autor}
    \label{fig_circintegrado}
\end{figure}

\begin{figure}[!h]
 \centering
    \caption{Circuito Integrado utilizado para o Receptor Óptico particionado} 
    \includegraphics[scale=0.4]{Resultados/Imagens/Image_CircuitoIntegrado.png}
    \legend{Fonte: Top View desenvolvido por Daniel Carvalho Lott}
    \label{fig_circintegrado_division}
    \nota{¹ Projeto desenvolvido por Victor Rodrigues Barbosa \cite{VictorRodrigues}\\² Projeto desenvolvido por Lucas Martins Chaves \cite{LucasChaves}}
\end{figure}

A \autoref{layoutcompleto} apresenta a implementação completa do Receptor Óptico projetado. A \autoref{layoutcompleto_division} mostra a mesma figura explicitando o que representa as diferentes partes do circuito. O projeto do Receptor Óptico ocupa uma área total de 422 633,9x666,96 $\mu$m² (~0,423 mm²).

\begin{figure}[!h]
 \centering
    \caption{Layout completo do circuito desenvolvido} 
    \includegraphics[scale=1, angle = 90]{Resultados/Imagens/Circuito Completo.png}
    \legend{Fonte: Produzido pelo autor}
    \label{layoutcompleto}
    \nota{Imagem rotacionada em 90° em sentido anti-horário}
\end{figure}

\begin{figure}[!h]
 \centering
    \caption{Layout completo do circuito particionado} 
    \includegraphics[scale=0.4]{Resultados/Imagens/Image_CircuitoCompleto.png}
    \legend{Fonte: Produzido pelo autor}
    \label{layoutcompleto_division}
\end{figure}

\begin{table}[]
\caption{Pinos presentes no circuito integrado para o Receptor Óptico}
\footnotesize
\begin{tabular}{cccll}
\toprule
\begin{tabular}[c]{@{}c@{}}Pino\\ Circuito \\ Integrado\end{tabular} & \begin{tabular}[c]{@{}c@{}}Pino\\ Chip\end{tabular} & Nome              & \multicolumn{1}{c}{Descrição}                                                                           & \multicolumn{1}{c}{Observação}                                                                                           \\
\midrule \midrule
1                                                                 & 40                                                  & vref\_pixel       & Não utilizado                                                                    &                                \\\midrule
2                                                                 & 41                                                  & vout\_clk         & Tensão de saída de relógio gerada                                                                       &                                \\\midrule
3                                                                 & 42                                                  & out\_dig\_blue    & \begin{tabular}[c]{@{}l@{}}Sinal de tensão digital\\ para cor azul\end{tabular}                         &                                \\\midrule
4                                                                 & 43                                                  & out\_dig\_green   & \begin{tabular}[c]{@{}l@{}}Sinal de tensão digital\\ para cor verde\end{tabular}                        &                                \\\midrule
5                                                                 & 44                                                  & VSS               & Terra                                                                                                   &                                \\\midrule
6                                                                 & 1                                                   & out\_dig\_red     & \begin{tabular}[c]{@{}l@{}}Sinal de tensão digital\\ para cor vermelha\end{tabular}                     &                                \\\midrule
7                                                                 & 2                                                   & iref\_test        & Não utilizado                                                                        &                                \\\midrule
23                                                                & 18                                                  & reset\_b\_blue    & \begin{tabular}[c]{@{}l@{}}Sinal de tensão de RESET\\ no APS para cor azul\end{tabular}                 & Ativo em nível baixo           \\\midrule
24                                                                & 19                                                  & reset\_b\_green   & \begin{tabular}[c]{@{}l@{}}Sinal de tensão de RESET\\ no APS para cor verde\end{tabular}                & Ativo em nível baixo           \\\midrule
25                                                                & 20                                                  & reset\_b\_red     & \begin{tabular}[c]{@{}l@{}}Sinal de tensão de RESET\\ no APS para cor vermelha\end{tabular}             & Ativo em nível baixo           \\\midrule
26                                                                & 21                                                  & tx\_blue          & \begin{tabular}[c]{@{}l@{}}Sinal de tensão de ENABLE\\ no APS para cor azul\end{tabular}                & Ativo em nível alto            \\\midrule
27                                                                & 22                                                  & tx\_green         & \begin{tabular}[c]{@{}l@{}}Sinal de tensão de ENABLE\\ no APS para cor verde\end{tabular}               & Ativo em nível alto            \\\midrule
28                                                                & 23                                                  & V18               & Tensão de alimentação de 1,8V                                                                           &                                \\\midrule
29                                                                & 24                                                  & VSS               & Terra                                                                                                   &                                \\\midrule
30                                                                & 25                                                  & tx\_red           & \begin{tabular}[c]{@{}l@{}}Sinal de tensão de ENABLE\\ no APS para cor vermelha\end{tabular}            & Ativo em nível alto            \\\midrule
31                                                                & 26                                                  & vdiode\_test      & \begin{tabular}[c]{@{}l@{}}Corrente que simula fotogeração\\ no APS de teste\end{tabular} &                                \\\midrule
32                                                                & 27                                                  & vdiode\_clk\_test & \begin{tabular}[c]{@{}l@{}}Corrente que simula fotogeração\\ no fotodiodo no TIA de teste\end{tabular} &                                \\\midrule
33                                                                & 28                                                  & out\_ana\_test    & \begin{tabular}[c]{@{}l@{}}Sinal de tensão analógico para o\\ APS de teste\end{tabular}                 &         \\\bottomrule                      
\end{tabular}
\label{tab_clcc44}
\legend{Fonte: Produzido pelo autor}
\end{table}

O bloco \textit{APS\_digitalized} projetado é apresentado na figura \autoref{layoutAPSDIG}. A \autoref{layoutAPSDIG_division} mostra a mesma figura explicitando o que representa as diferentes parte do circuito. O projeto do bloco ocupa uma área total aproximada de 3307 $\mu$m².

\begin{figure}[!h]
 \centering
    \centering
    \caption{Layout do bloco \textit{APS\_digitalized}} 
    \includegraphics[scale=0.8]{Resultados/Imagens/APS_DIGITALIZED.png}
    \legend{Fonte: Produzido pelo autor}
    \label{layoutAPSDIG}
\end{figure}

\begin{figure}[!h]
 \centering
    \centering
    \caption{Layout do bloco \textit{APS\_digitalized} particionado} 
    \includegraphics[scale=0.3]{Resultados/Imagens/Image_APS_Digitalized.png}
    \legend{Fonte: Produzido pelo autor}
    \label{layoutAPSDIG_division}
\end{figure}

O bloco \textit{APS\_clk} projetado é apresentado na figura \autoref{layoutTIA}. A \autoref{layoutTIA_division} mostra a mesma figura explicitando o que representa as diferentes parte do circuito. O projeto do bloco ocupa uma área total aproximada de 102107 um².

\begin{figure}[!h]
 \centering
    \begin{minipage}{0.5\textwidth}
    \centering
    \caption{Layout do bloco \textit{APS\_clk}} 
    \includegraphics[scale=0.7]{Resultados/Imagens/TIA.png}
    \legend{Fonte: Produzido pelo autor}
    \label{layoutTIA}
    \end{minipage}
    \hfill
    \begin{minipage}{0.4\textwidth}
    \centering
    \caption{Layout do bloco \textit{APS\_clk} particionado}
    \includegraphics[scale=0.4]{Resultados/Imagens/Image_TIA.png}
    \legend{Fonte: Produzido pelo autor}
    \label{layoutTIA_division}
    \end{minipage}
\end{figure}

O bloco \textit{APS\_3} projetado é apresentado na figura \autoref{layoutAPS_3}. A \autoref{layoutAPS_3_division} mostra a mesma figura explicitando o que representa as diferentes parte do circuito. O projeto do bloco ocupa uma área aproximada de 123353 $\mu$m².

\begin{figure}[!h]
    \centering
    \caption{Layout do bloco \textit{APS\_3}} 
    \includegraphics[scale=1]{Resultados/Imagens/APS_3.png}
    \legend{Fonte: Produzido pelo autor}
    \label{layoutAPS_3}
\end{figure}

\begin{figure}[!h]
    \centering
    \caption{Layout do bloco \textit{APS\_3} particionado} 
    \includegraphics[scale=0.4]{Resultados/Imagens/Image_APS_3.png}
    \legend{Fonte: Produzido pelo autor}
    \label{layoutAPS_3_division}
\end{figure}


% ----------------------------------------------------------
% PARTE
% ----------------------------------------------------------

% ----------------------------------------------------------
% Finaliza a parte no bookmark do PDF
% para que se inicie o bookmark na raiz
% e adiciona espaço de parte no Sumário
% ----------------------------------------------------------
\phantompart

% ---
% Conclusão
% ---
\chapter{Conclusão}

O autor do trabalho apresentou a base teórica dos dispositivos APS e TIA, além das especificações de um projeto de Receptor Óptico, as soluções propostas pelo autor para o seu desenvolvimento. as simulações necessárias para avaliação da proposta, e a elaboração do layout dos principais circuitos dentro do trabalho.

O projeto como um todo necessitou de um esforço em conjunto de várias pessoas para que pudesse ser realizado, que foram citadas ao longo do trabalho assim que apresentados suas atividades.

Diversos desafios apareceram ao longo de todo o trabalho, dos quais necessitam de ampla pesquisa acadêmica e também orientação para o devido entendimento dos problemas que apareceram.

O trabalho como um todo teve seus objetivos alcançados, mostrando o desenvolvimento e simulação de um projeto de Receptor Óptico para três cores, e seu circuito integrado foi desenvolvido para que seja realizado a sua validação.

\subsection{Trabalhos futuros}

O autor propõe que com a fabricação do circuito integrado, sejam realizadas todas as medições para validação do trabalho, e que este possa servir de conhecimento para outros que se proporem em estender ao que foi aqui desenvolvido.

Para as simulações, algumas estimativas foram realizadas pelo autor, que com as amostras do CI podem ser aprimoradas, com foco em trabalhos futuros. Simulações mais precisas de forma a gerar modelos e circuitos mais complexos podem ser realizados com base no que foi aqui apresentado.
% ---

% ----------------------------------------------------------
% ELEMENTOS PÓS-TEXTUAIS
% ----------------------------------------------------------
\postextual
% ----------------------------------------------------------

% ----------------------------------------------------------
% Referências bibliográficas
% ----------------------------------------------------------
\bibliography{Referencias}


% ---
% Inicia os anexos
% ---
\begin{anexosenv}

% Imprime uma página indicando o início dos anexos
\partanexos

\chapter{Blocos Adicionais}

\input{Anexos/Circuitos/CircuitosAdicionais}
\section{Espelho de Corrente}
\label{anexoespelhos}

\subsection{Espelho de Corrente NMOS}

Um espelho de corrente \'e um circuito que replica o sinal de uma corrente de refer\^encia em outras sa\'idas, podendo ser multiplicado por um fator de ajuste. A \autoref{fig_espelho} mostra a representa{\c c}\~ao de um circuito com tal caracter\'istica, utilizando transistores NMOS. O espelho de corrente NMOS tamb\'em \'e chamado de dreno de corrente, por drenar a corrente nos ramos.

\begin{figure}[htb]
    \label{fig_espelho}
    \centering
    \caption{Espelho de corrente NMOS} 
    \includegraphics[scale=0.4]{Circuitos/current_mirror_example.png}
    \legend{Fonte: Produzido pelo autor}
\end{figure}

Nesta figura, est\'a representado a utiliza{\c c}\~ao de uma corrente de refer\^encia Iref para gerar as correntes Io1, Io2 at\'e Ion, onde "n" \'e o n\'umero de transistores se referenciando por \emph{Qref}.

No circuito demonstrado pela \autoref{fig_espelho}, o transistor \emph{Qref} tem func\^ao de captar a corrente igual \'a \emph{Iref}, para servir de refer\^encia aos outros transistores \emph{Q1}, \emph{Q2} at\'e \emph{Qn}, onde \emph{n} \'e o numero de transistores que se referenciam de \emph{Qref}, e que s\~ao chamados de bra{\c c}os do espelho. O potencial \emph{VDD\_IREF} \'e um valor de tens\~o utilizado para polarizar Iref, que n\~ao precisa ser igual ao valor de alimenta{\c c}\~ao dos outros bra{\c c}os.

Dado os parâmetrosWx/Lnde cada transistor, onde "x" é o número indicado do transistor, podemos calcular o valor de corrente de cada braço utilizando a fórmula apresentada na \autoref{eq_espcorpmos}.

\begin{equation}
    \label{eq_espcor}
    I_{OX} = Iref\frac{W_x/L_x}{W_{ref}/L_{ref}}
\end{equation}

Onde $I{ox}$ \'e a corrente de entrada do bra{\c c}o \emph{Qx}. 

Para que esse circuito funcione devidamente em cada sa\'ida, cada bra{\c c}o do espelho deve estar necessariamente operando na regi\~ao ativa, pois a \autoref{eq_espcor} \'e deduzida levando isso em conta. Para que isso aconte{\c c}a, devemos respeitar a \autoref{eq_curmirror_req} \cite{RazaviFundM}.

\begin{equation}
    \label{eq_curmirror_req}
    v_{DS} \geq v_{GS} - V_t
\end{equation}

Onde:

\begin{itemize}
    \item $v_{DS}$ \'e a tens\~ao entre o dreno e fonte do transistor
    \item $v_{GS}$ \'e a tens\~ao entre o porta e fonte do transistor
    \item $V_{t}$ \'e a tens\~ao de limiar do transistor
\end{itemize}

\subsection{Espelho de Corrente PMOS}

Um circuito de espelho de corrente pode ser constru\'ido com transistores PMOS, utilizando o mesmo racioc\'inio de constru{\c c}\~ao do circuito NMOS, conforme a \autoref{fig_curmir_pmos}. A diferen{\c c}a principal \'e que em vez de ser um dreno de corrente, o PMOS ser\'a um fornecedor de corrente.

\begin{figure}[htb]
    \label{fig_curmir_pmos}
    \centering
    \caption{Espelho de corrente PMOS} 
    \includegraphics[scale=0.4]{Circuitos/current_mirror_example_pmos.png}
    \legend{Fonte: Produzido pelo autor}
\end{figure}

O funcionamento do circuito PMOS segue o mesmo principio do NMOS, porém, ao inv\'es de receber corrente fixada a um n\'o, ele fornece. O potencial \emph{negVDD\_IREF} \'e um valor de tens\~o utilizado para polarizar Iref, que n\~ao precisa ser igual ao valor de alimenta{\c c}\~ao negativa/terra dos outros bra{\c c}os.

Dado os parâmetrosWx/Lnde cada transistor, onde "x" é o número indicado do transistor, podemos calcular o valor de corrente de cada braço utilizando a fórmula apresentada na \autoref{eq_espcorpmos}.

\begin{equation}
    \label{eq_espcorpmos}
    I_{OX} = I_{ref}\frac{W_x/L_x}{W_{ref}/L_{ref}}
\end{equation}

Onde $I_{OX}$ \'e a corrente de sa\'ida do bra{\c c}o \emph{Qx}. 

Assim como no caso do NMOS, todos transistores devem estar na regi\~ao ativa, e respeitar a seguinte \autoref{eq_curmirror_reqpmos} \cite{RazaviFundM}.

\begin{equation}
    \label{eq_curmirror_reqpmos}
    v_{SD} \geq v_{SG} - |V_t|
\end{equation}

Onde:

\begin{itemize}
    \item $v_{SD}$ \'e a tens\~ao entre o fonte e dreno do transistor
    \item $v_{SG}$ \'e a tens\~ao entre a fonte e oirta do transistor
    \item $V_{t}$ \'e a tens\~ao de limiar do transistor
\end{itemize}



\end{anexosenv}

% ----------------------------------------------------------
% Glossário
% ----------------------------------------------------------
%
% Consulte o manual da classe abntex2 para orientações sobre o glossário.
%
%\glossary

%---------------------------------------------------------------------
% INDICE REMISSIVO
%---------------------------------------------------------------------
\phantompart
\printindex
%---------------------------------------------------------------------

\end{document}
