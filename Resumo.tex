% ---
% RESUMOS
% ---

% resumo em português
\setlength{\absparsep}{18pt} % ajusta o espaçamento dos parágrafos do resumo
\begin{resumo}

O trabalho explora a implementação de Optoreceptores utilizando uma tecnologia CMOS de 180 nm da TSMC. São projetados um pixel do tipo APS (Active Sensor Pixel) e
um Amplificador de Transimpedância (TIA). O projeto completo do optoreceptor contém quatro instâncias do pixel APS e uma instância do TIA. Além disso, para permitir a avaliação do sistema proposto, um pixel APS e um TIA adicionais foram incluídos no chip.

Três APS’s t\^em a função de captação de uma cor distinta (Azul, Verde ou Vermelho) de uma fonte luminosa. O circuito desenvolvido é feito de maneira idêntica para as 3 cores, e um filtro de luz externo é utilizado para que seja captada apenas a cor de interesse. O APS é constituído por um fotodiodo quadrado de dimensões 25 x 25 µm (625 µm²), e um
circuito para conversão da corrente fotogerada para uma tensão de saída.. Na saída do APS, o sinal é comparado à uma tensão de referência durante sua digitalização. A tensão de referência é programável, permitindo flexibilidade na digitalização de diferentes intensidades de corrente fotogerada. Um APS é utilizado apenas para fins de teste e seu projeto elétrico é igual aos demais, com exceção de um pino extra do qual pode ser drenada uma corrente elétrica que emula o efeito fotocondutivo, sem a presença da luz.

O TIA tem a função de extrair um sinal de referência de relógio a partir de uma fonte luminosa, e servir como referência temporal para os outros circuitos, por exemplo, os pixels APS’s. O fotodiodo utilizado tem dimensão de 25x25 µm (625 µm²). O circuito foi projetado de forma a permitir uma frequência nominal de operação de 100 kHz. Assim como o APS de teste, existe um segundo TIA com um pino extra para simular uma fotocorrente sem a pŕesença de luz.

O layout de diversos circuitos analógicos e de sinais mistos auxiliares para a operação do APS e do TIA foram também projetados. O circuito integrado completo ocupa uma área de 1,6x1,6 mm (2,56 mm²), pois também contém um oscilador de tensão, sensor de temperatura, e acionadores de microLEDS. O projeto foi submetido para fabricação em setembro de 2020, e sua caracterização será possível a partir de março de 2021.


 \textbf{Palavras-chave}: APS. CMOS. FOTODIODO. PIXEL. TIA.
\end{resumo}