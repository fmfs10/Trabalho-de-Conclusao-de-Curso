
% resumo em inglês
\begin{resumo}[Abstract]
 \begin{otherlanguage*}{english}

This work explores the implementation of Optoreceptors using a \textit{CMOS TSMC 180 nm} technology from TSMC. An APS (Active Sensor Pixel) and a Transimpedance Amplifier (TIA) are proposed. Three APSs have the function of capturing different colors (Blue, Green or Red) from a light source. The developed circuit is made identically for the 3 colors, and an external light filter is used so that only the color of interest is captured. The APS consists of a square photodiode of dimensions 25x25$\mu$m (625$\mu$m²), associated with a circuit for converting the photogenerated current to an output voltage. At the APS output, the signal is compared to a reference voltage during its digitalization. Reference voltage is programmable, allowing flexibility in digitalizing different intensities of photogenerated current. An APS is used only for testing purposes and its electrical design is the same as the others, with the addition of an extra pin for injecting an electric current emulating photogeneration, without the presence of light. The TIA has the function of extracting a clock reference signal from a light source, at the same time as it serves as a time reference for other circuits, for example, APSs pixels. The photodiode used has a dimension of 25x25$\mu$m (625$\mu$m²). The circuit was designed to allow a nominal operating frequency of 100 kHz. Like the test APS, there is a second TIA with an extra pin to simulate a photocurrent without the presence of light. The design of several analog circuits and mixed auxiliary signals for the functioning of the APS and TIA were also carried out. The complete integrated circuit occupies an area of 1.6x1.6 mm (2.56 mm²), as it also contains a voltage oscillator, temperature sensor, and microLEDS actuators. The project was submitted for production in September 2020, and its characterization will be possible from March 2021.

   \vspace{\onelineskip}
 
   \noindent 
   \textbf{Keywords}: APS. CMOS. PHOTODIODE. TIA.
 \end{otherlanguage*}
\end{resumo}