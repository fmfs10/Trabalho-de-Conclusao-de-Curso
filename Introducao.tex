% ----------------------------------------------------------
% Introdução (exemplo de capítulo sem numeração, mas presente no Sumário)
% ----------------------------------------------------------
\chapter[Introdução]{Introdução}
%$\addcontentsline{toc}{chapter}{Introdução}
% ----------------------------------------------------------
Receptores ópticos têm uma vasta gama de aplicações. Podemos utilizá-los para detectar a presença ou ausência de luz no ambiente em determinada faixa de frequência; recepção de sinais luminosos de diferentes cores para construir uma foto em uma câmera; recepção de sinais codificados em luz para realizar uma transmissão de dados sem fio (Li-Fi); detecção de temperatura em um objeto via a sua emissão de luz infravermelha; entre incontáveis outras utilidades. A forma como o circuito do sensor é desenvolvido varia muito com a aplicação, e diversas topologias podem ser utilizadas. Mesmo dentro de uma classe específica de sensores, diversas variações de circuitos podem existir.

Neste trabalho, é desenvolvido o projeto eletrônico dos principais blocos analógicos de um Receptor \'Optico integrado em tecnologia CMOS 180 nm. Para realizar a recepção dos dados, um pixel ativo do tipo APS (\textit{Active Pixel Sensor}) e um TIA (\textit{Transimpedance Amplifier}) para captação de refer\^encia de temporização foram projetados. O APS e o TIA são topologias muito estudadas e utilizadas, devido principalmente à facilidade com que podem ser integrados em um CI utilizando tecnologia CMOS, sua facilidade de acoplamento com outros circuitos do mesmo CI, e também pelo baixo custo por pixel em comparação com outros tipos de Receptores Ópticos. 

Junto à construção dos blocos citados, diversos outros circuitos são necessários para que se permita o condicionamento e processamento do sinal luminoso transferido para o domínio elétrico. Dentre os circuitos auxiliares que compõem todo um sistema de Pixel Ativo, podemos citar Amplificadores Operacionais, utilizados para amplificar o sinal gerado pelo sensor; circuitos geradores de temporização, que definem o momento de captação de luz do pixel e momento este poderá ser operado; e circuitos de controle que devem ser utilizados para garantir sincronia entre todos elementos do sistema

O desenvolvimento dos principais circuitos analógicos de um Receptor Óptico integrado serão apresentados, desde o seu detalhamento teórico, concepção do circuito e subcircuitos necessários ao sistema, desenvolvimento de esquemático para realização de simulações iniciais, construção do layout de circuito integrado, desenvolvimento de um CI para permitir a ligação dos sinais e realização de testes, e finalmente, simulações para validação de todo sistema. A tecnologia utilizada no processo é a \textit{CMOS TSMC 180nm}.

Todo o trabalho é desenvolvido com apoio do OptMA\textsuperscript{Lab} , laboratório que se apresenta internamente à Escola de Engenharia da UFMG, e que possui o ferramental necessário para permitir a realização de todos o procedimentos aqui citados, com exceção da fabricação do APS e seu encapsulamento, que é realizada em local externo por meio de uma parceria da universidade com a IMEC (\textit{Interuniversity Microelectronics Centre}).

\section{Motivação e Objetivos do Trabalho}

Dado a infinidade de aplicações do qual um APS pode ser utilizado, a fundamentação de parâmetros de referência para desenvolvimento de projetos de circuitos que se faz presente é de essencial importância para possibilitar a escolha de qual projeto é mais adequado à aplicação. Diferentes figuras de mérito podem ser avaliadas para otimizar uma ou mais características de interesse técnico e/ou econômico do projeto.

O trabalho tem como principal objetivo o desenvolvimento de um circuito de Sensor de Pixel Ativo, e um TIA, tomando como referência trabalhos j\'a desenvolvidos por outros autores. O desenvolvimento apresenta o projeto elétrico, simulação, e desenvolvimento do layout do CI. 

São desenvolvidos seis circuitos principais, sendo três com a função de detectar diferentes cores (azul, verde e amarelo) utilizando um APS, um circuito extrator de sinal de relógio de uma fonte luminosa atrav\'es de um TIA, e um APS e um TIA para fins de testes, dos quais são utilizados para testar as funcionalidades dos circuitos citados sem a necessidade de uma fonte luminosa. 

O dispositivo projetado deve se adequar a todas as características recomendáveis de um projeto de circuito analógico integrado CMOS, devendo ser avaliado e otimizado para aprimorar certas propriedades importantes de layout como perdas ohmicas, capacitâncias parasitas, ruídos e consumo.

Alguns circuitos auxiliares compõem conjuntamente o projeto, como Amplificadores Operacionais, Filtros e Multiplexadores, que são elementos comuns em projetos de APS’s e TIA's, utilizados para transferir e controlar a informação gerada pelo sensor, mas que não são o principal foco de análise.

\section{Estrutura do Trabalho}

A monografia apresenta o desenvolvimento de um projeto de APS's e TIA's com base em algumas figuras de mérito comumente utilizadas na indústria.. O texto foi dividido em quatro capítulos, além desta introdução, sendo eles Revisão Bibliográfica, Metodologia, Resultados e Conclusão.

Revisão Bibliográfica, capítulo 2 deste trabalho, visa apresentar um resumo histórico simplificado sobre Receptores Ópticos, conceituação de elementos chave para o entendimento do que é apresentado em capítulos posteriores, citação de outros trabalhos relacionados ao tema aprovados pela comunidade científica e detalhamento do trabalho base que foi tomado como principal inspiração da monografia.
Metodologia, capítulo 3 deste trabalho, visa apresentar o processo adotado para desenvolvimento do circuito de interesse, com base no trabalho de inspiração citado, demonstrando o processo de projeto à nível de esquemático elétrico, simulações, avaliações de figuras de mérito em ambiente simulado e o design do layout do circuito analógico.

Resultados, capítulo 4 deste trabalho, apresenta todos resultados de simulação obtidos de acordo com o modelo metodológico previamente apresentado, e a comprovação de atendimento aos objetivos tais quais formulados. Uma série de imagens, gráficos e dados são apresentados de forma a permitir a avaliação do projeto como um todo.

Conclusão, capítulo 5 deste trabalho, retoma os resultados e avalia se os objetivos foram devidamente alcançados. Sugestões de trabalhos posteriores são apresentados para explorar alguns tópicos que o autor entende que possam complementar todo o estudo aqui realizado.

\section{Padronização de identifição dos elementos dos circuitos}
\label{section:padrao_sinais}

Para facilitar a demonstração dos circuitos presentes neste trabalho, o autor padronizou um conjunto de cores que estarão presentes em cada elemento ilustrados para cada circuito referenciado:

\begin{itemize}
    \item \textbf{\color{red}VERMELHO}: indica o nome de uma inst\^ancia de um componente
    \item \textbf{\color{orange}LARANJA}: indica o nome de um sinal interno
    \item \textbf{\color{green}VERDE}: indica um sinal de entrada
    \item \textbf{\color{blue}AZUL}: indica um sinal de sa\'ida
    \item \textbf{\color{gray}CINZA}: indica o nome de um sinal bidirecional (entrada e/ou sa\'ida)
\end{itemize}